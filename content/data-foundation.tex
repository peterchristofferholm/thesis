\chapter{Overview of Data Resources}

Data stands as the cornerstone of both precision medicine and machine learning.
Its availability is crucial; without it, research in these
fields would be almost impossible. 
The emergence of high-throughput analyses and \acp{EHR} has 
led to a Cambrian explosion of the volume of data being generated,
which has the potential of revolutionizing the 
entire landscape of biomedical research. 
However, the sensitive nature of this data means
that access is frequently a major challenge, 
often serving as a major bottleneck in
many research endeavors.

In the context of the studies conducted for this thesis, I have been in the
privileged position of working within a research group where permissions, 
data access, and the necessary infrastructure were already well-established.
In terms of infrastructure, a key aspect of our data handling involved the use 
of a secure high-performance computing environment. 
This not only ensured the efficient processing of large datasets and 
training of large neural networks, 
but also maintained the highest standards of data security and
confidentiality, which are paramount in dealing with sensitive health records.
These aspects have been instrumental in enabling and driving 
the research and analyses presented in this thesis.

Focusing on the data itself,
this chapter aims to provide a comprehensive overview of the various data
sources utilized in the studies comprising this thesis. It details the
databases and registries that were accessed and analyzed,
highlighting how each contributed to the research. 

\section{The Danish Civil Registration System}

\Ac{CPR} is the central administrative register in Denmark,
and stores personal information on the entire Danish population,
including birth date, sex, addresses, vital status, 
and, importantly, a unique personal identification number, 
known as \enquote{\ac{CPR}-nummer} or \enquote{personnummer}.
~\autocite{schmidtDanish2014}
The \ac{CPR} number is assigned at birth or upon obtaining Danish citizenship, 
and have been in use since 1968.
As of January 2, 2014, 
\num{9484792} \ac{CPR} numbers were assigned.
~\autocite{schmidtDanish2014}
Of these \num{5685912} were \enquote{active},
and \num{3798880} were \enquote{non-active},
the latter primarily attributed to death and emigration.
~\autocite{schmidtDanish2014}

In Denmark, the \ac{CPR} number is as essential as a bicycle, 
required for opening a bank account, 
borrowing a library book, 
getting treated for appendicitis, and everything in between.
This widespread use, 
combined with the country's long-standing tradition
of organising and keeping record of detailed data in 
administrative databases and registries,
have enabled the construction of a large network
of interlinkable epidemiological resources.
~\autocite{schmidtDanish2019}
In this light, the whole nation can be utilized as a research cohort
as outlined by \textcite{frankEpidemiology2000} in a letter
to \textit{Science}.

\section{The Danish National Patient Register}

Of the Danish registries, \ac{LPR} is arguably one of the most important.
\ac{LPR}, or \enquote{Landspatientregisteret}, 
is a comprehensive clinical register that has
been instrumental for clinical research and administration in Denmark,
and serves multiple critical functions.
~\autocite{schmidtDanish2015}
Primarily, it underpins the Danish Health and Medicines Authority's hospital
statistics and is a main foundation for health economic calculations.
Additionally, the \ac{LPR} is instrumental in monitoring the
prevalence of various diseases and treatments. 
Furthermore, the registry plays a key role in
facilitating quality assurance of Danish healthcare services and provides
hospital physicians access to patients' hospitalization histories, enhancing
patient care and treatment efficacy.
~\autocite{schmidtDanish2015}
The register is updated monthly based on reports from the hospitals,
and has been collecting data continuously since 1977.
~\autocite{schmidtDanish2015}

\subsection{Content and Structure}

The \ac{LPR} encompasses a wide array of data on each individual, 
such as personal information, admission and discharge details, diagnoses, 
examinations, treatments including surgeries, information on accidents, 
and additional details concerning births. 
~\autocite{schmidtDanish2015}
The information in the register is organised in a
structured format with different data types being stored 
in distinct tables that can be linked following a 
specified relational data model.
~\autocite{lpr2dok}

This data model, referred to as \acsfont{LPR2},  
have remained largely unchanged since the release of the registry in 1977.
However, in early 2019, it underwent a significant overhaul to a 
new and refined data model, \acsfont{LPR3}.
~\autocite{nielsenLPR32018}
The \acsfont{LPR3} model addresses certain limitations of its predecessor, 
notably enabling the creation of more fine-grained patient care timelines. 
While the specifics of these improvements are beyond the scope of this thesis, 
it is important to note that the \acsfont{LPR3} model is not entirely 
backwards compatible with the \acsfont{LPR2} data model. 
Nevertheless, depending on the specific use case, 
it remains possible to create data extracts that are compatible
to one another.

\subsection{Classification of Diseases}

The highly structured data within the \ac{LPR} 
is coded using the national \acs{SKS} classification scheme (\enquote{\acl{SKS}}), 
a collection of Danish, international, and Nordic classification standards
maintained by the Danish Health Data Authority.
~\autocite{schmidtDanish2015}
These standards includes
the \acfi{NOMESCO} system for surgical procedures,
the \acfi{ATC} system for medication,
and the \acfi{ICD} system for diagnoses. 
~\autocite{schmidtDanish2015}

Focusing on diagnosis codes, 
the \ac{LPR} is currently using the \ac{ICD-10},
and have been doing so since the start of 1994
where it replaced the \acsu{ICD-8}.
~\autocite{schmidtDanish2015}
This transition can complicate longitudinal studies of disease occurence, 
but prior efforts by the Brunak group have succesfully
created a mapping between \ac{ICD-8} and \ac{ICD-10}
that can be used to mitigate such challenges.
~\autocite{pedersenUnidirectional2023}

The \ac{ICD-10} coding system follows a hierarchical structure 
with every code beginning with a letter followed by two or more digits.
Each code falls in one of 21 high-level categorisation of diagnoses 
that e.g.  includes chapters
(ii)~\emph{Neoplasms};
(iv)~\emph{Endocrine, Nutritional, and Metabolic Diseases};
and (ix)~\emph{Diseases of the Circulatory System}.
Using the latter as an example, 
\cref{fig:icd10-hierarchy} shows the hierarchical structure of the
\ac{ICD-10}.

% figure: icd10-hierarchy→
\begin{figure}
\begin{tikzpicture}[
    every node/.append style = {
        draw, anchor = west, 
        minimum width=10mm, 
        minimum height=4mm,
        font=\tlfstyle\scriptsize,
        rounded corners=1pt,
        fill=color2!5
    },
    sel/.style = {fill=color2!20, draw=black, thick},
    txt/.style = {
        fill=none, draw=none, font=\footnotesize, text width=4.0cm,
        text height=5mm
    },
    grow via three points={
        one child    at (1.0, -0.5) and 
        two children at (1.0, -0.5) 
                    and (1.0, -1.0)
    },
    edge from parent path={
        (\tikzparentnode.east) 
        -| ([xshift=-4.2]\tikzchildnode.west)
        |- (\tikzchildnode.west)
    }]
    \node[sel] {chap. IX}
        child {node {I00-I02}}
        child {node {I05-I09}}
        child {node {I10-I15}}
        child {node[sel] {I20-I25}
            child {node (i20) {I20}}
            child {node [sel] (i21) {I21}
                child {node (1) {I21.0}}
                child {node (2) {I21.1}}
                child {node (3) {I21.3}}
                child {node (4) {I21.4}}
                child {node (5) {I21.9}}
            }
            child {node {I22}}
            child {node {I24}}
            child {node {I25}}
        }
        child {node {I26-I28}}
        child {node {I30-I52}}
        child {node {I60-I69}}
        child {node {I70-I79}}
        child {node {I80-I89}}
        child {node {I95-I99}}
    ;

    \node [txt] (t1) at (6.3cm, -1.5cm) 
        {acute transmural myocardial infarction of anterior wall};
    \node [txt, below=3mm of t1.center] (t2) 
        {acute transmural myocardial infarction of inferior wall};
    \node [txt, below=3mm of t2.center] (t3) 
        {acute transmural myocardial infarction not otherwise specified};
    \node [txt, below=3mm of t3.center] (t4) 
        {acute myocardial infarction with non-ST elevation};
    \node [txt, below=3mm of t4.center] (t5) 
        {acute myocardial infarction not otherwise specified};

    \draw (1.east) .. controls +(2ex,0) and +(-3ex,0) .. (t1.west);
    \draw (2.east) .. controls +(2ex,0) and +(-3ex,0) .. (t2.west);
    \draw (3.east) .. controls +(2ex,0) and +(-3ex,0) .. (t3.west);
    \draw (4.east) .. controls +(2ex,0) and +(-3ex,0) .. (t4.west);
    \draw (5.east) .. controls +(2ex,0) and +(-2ex,0) .. (t5.west);
    
\end{tikzpicture}
\caption[The ICD-10 Hierarchy]{%
    Schematic illustrating the hierarchical structure of the \ac{ICD-10} coding
    system. 
    Using Chapter IX---which covers diseases of the circulatory system
    (codes I00-I99) and comprises ten different \enquote{blocks}---as an example, 
    the diagram shows how these blocks are segmented into 
    \enquote{level 3} codes, such as I20 to I25, 
    corresponding to ischemic heart disease. 
    It provides an expanded view of the I21 category, 
    containing specific subtypes, \enquote{level 4} codes, 
    of acute myocardial infarctions 
    from I21.0 for the anterior wall 
    to I21.9 for unspecified instances. 
    Deeper levels of the hierarchy also exists,
    but is not included in this diagram.
}
\label{fig:icd10-hierarchy}
\end{figure}
% ←

\section{The East Danish Heart Registry}

\ac{RKKP} is a program set in place to support the
management and infrastructure related to clinical quality databases.


Data from some of the other general registries, 
including the \ac{LPR}, is being used to populate 
specific variables in different \ac{RKKP} registries.

The \ac{RKKP} undergoes revision by the National Health Authority
every three years to continuously assess specific 
criteria for function, safety, and methodology.


\section{The Register of Pharmaceutical Sales}



\section{The Causes of Death Register}

When a person in Denmark dies, a doctor performs a post-mortem examination. The
doctor fills in a death certificate and reports the death by submitting the
certificate to the Danish Health Authority.

\section{The BigTempHealth Project}

\subsection{Patient Notes}

The DNPR does not contain information on tobacco usage, blood pressure, 
body-mass index, and other important covariates to consider in the
context of cardiovascular diseases.

% figure: journal text→
\begin{figure}
{
% Define new commands for each category
\newcommand{\cbox}[2]{%
    \tcbox[on line, colback=#1!50!white, boxsep=1.5pt, colframe=black,
           valign=center, left=0pt, right=0pt, top=0pt, bottom=0pt, boxrule=1pt]{#2}%
}%
\newcommand{\diag}[1]{\cbox{color3}{#1}}
\newcommand{\symp}[1]{\cbox{color4}{#1}}
\newcommand{\drug}[1]{\cbox{color5}{#1}}
\newcommand{\quan}[1]{\cbox{color6}{#1}}
\newcommand{\kwpa}[1]{\cbox{color7}{#1}}

\begin{Verbatim}[commandchars=\\\{\}, fontsize=\scriptsize, 
    frame=single, framesep=1em, numbers=left, numbersep=3pt]
68-year-old woman, no earlier known cardiovascular disease, 
referred by gp for observation of \diag{angina pectoris}.

Risk factors:
\diag{Hypertension}: Yes, newly discovered, currently well-treated.
\diag{Hypercholesterolemia}: Yes. Under treatment.
Family: No family history of \diag{ischemic heart disease}.
Smoking: Quit 15 years ago, before that 20 pack-years.
Claudication: No.

Previously:
Known since 2001 with \diag{type II diabetes mellitus}, treated 
with \drug{Metformin}.  Followed up with regular checks. 
Complications with discrete neuropathy and beginning 
macular degeneration.

Currently:
For about 3 months, \symp{intermittent pressure in the left side}
\symp{of the chest}. No radiation, independent of physical 
activity. Daily attacks lasting a few seconds. During the 
same period, has been diagnosed with \diag{severely elevated blood}
\diag{pressure}. Recently started anti-hypertensive treatment with
good effect.

Medication:
- \drug{Metformin} \quan{1000 mg} x 2
- \drug{Simvastatin} \quan{20 mg} x 1
...

Objective:
Normal general condition.
Nutritional status above average.
\kwpa{Blood pressure 149/91}, \kwpa{pulse 76}, \kwpa{height 177 cm}, \kwpa{weight 96 kg}
\end{Verbatim}
}
\caption{Artificial example of unstructured text in an \ac{EHR}}
\end{figure}% ←
