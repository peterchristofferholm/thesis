\chapter{Health Informatics}

Health informatics is a cross-disciplinary field
that aims at developing methods and techniques
for the study of healthcare data.

It operates in the intersection between medicine and computer science,
and involves disciplines such as
bioinformatics, software engineering, statistics, information systems,
data science, and artificial intelligence.

The goal of health informatics is the data-driven improvement
of delivery and accuracy of healthcare to the individual patient.
    
An electronic health record
is a systematized collection of health data 
stored in a digital format.
Electronic health records may include data such as
medical history, drug prescriptions, laboratory test results,
radiology images, body measurements, and billing information.
EHRs enable following patients in both time and space;
not only in the physical space that is the cardiology department
and the intensive care unit, but also in the health-and-disease space.

\begin{itemize}
    \item \textbf{Decision support systems}
    \item \textbf{Personalized medicine}
    \item \textbf{Mobile health applications} (aka. \textit{mHealth})
    \item \textbf{Screening}
    \item \textbf{Surveillance}
    \item \textbf{Medical outcomes}
\end{itemize}

\marginnote{
    The five Vs of big data:
    \begin{itemize}
        \item Volume
        \item Velocity
        \item Variety
        \item Value
        \item Veracity
    \end{itemize}
}

\section{Clinical decision support systems}

Deep learning can identify patterns in sparse, noisy,
and very heterogenous data without the need for expert feature engineering
\cite{norgeotCall2019}.



