\chapter{Thesis Objectives and Structure} 
\label{chap:overview}

With the overall aim of furthering our knowledge on \ac{IHD},
a leading global cause of morbidity and mortality,
this thesis explores the potential of \ac{ML} in deriving
clinically relevant insights from extensive electronic health data. 
The research primarily focuses on the development of \ac{ML}-based
methods for data-driven phenotyping and risk prediction,
contributing to the advancement of 
precision medicine in secondary prevention of \ac{IHD}. 

\section*{Thesis Objectives}

The primary objectives of the thesis are:

\begin{fullwidth}
\begin{multicols}{2}
\raggedcolumns

\begin{enumerate}
    \item From a comprehensive database including hospital data on
        over \num{2.6} million individuals with data 
        originating from electronic health records and
        national and clinical registries, extract and curate 
        high-quality data and setup \ac{ML} experiments and analyses.
        This includes:
    \begin{enumerate}
        \item Writing data-processing code to 
            consolidate, clean, and organize heterogeneous
            data from various sources.
        \item Ensuring the maintenance of robust scientific software 
            engineering practices, including version control, workflow
            managers, and containerization for reproducibility.
        \item Creating definition algorithms for the precise identification of 
            patient populations, disease onset, and clinical outcomes.
    \end{enumerate}
    \item Using unsupervised clustering, explore and characterise the 
        comorbidity landscape in \ac{IHD}, identifying distinct patternts
        of multimorbidity and their associated risk of disease progression
        and mortality.
    \item Develop and validate clinically relevant 
        risk-prediction algorithms for \ac{IHD} using using real-world 
        heterogenous healthcare data and right-censored time-to-event 
        outcomes.
    \begin{enumerate}
        \item As a proof-of-concept, start by focusing on prediction
            models for all-cause mortality, and endpoint that is easy to 
            define and where competing risks is of little concern.
        \item Expand the prediction targets to include cardiovascular
            mortality and specific markers of disease progression, 
            which requires accounting for competing risks.
    \end{enumerate}
    \item Use \ac{XAI} techniques, such as \ac{SHAP}-analysis, to deconvolve 
        the \ac{ML} prediction models such that it is possible to understand 
        the decision-making process of these models and to identify the key 
        factors influencing predictions. This will enhance the
        transparency and trustworthiness of the models for potential clinical 
        application.
\end{enumerate}

\end{multicols}
\end{fullwidth}

\newpage
\section*{Structure and Scope}

The thesis is written in the form of a synopsis 
and is as such based on three key manuscripts around 
which the content of thesis is centered.
The following gives an overview of the structure and 
provides a high-level outline of the included chapters.

\begin{itemize} 
    \item In \autoref{chap:precision-medicine},
        I provide some background on the pathophysiology and disease
        manifestations of \ac{IHD}, 
        motivating the role of precision medicine for 
        improved secondary prevention in this disease.

    \item In \autoref{chap:machine-learning}, 
        I introduce central concepts of machine learning, 
        with a specific focus on neural networks. 
        While generally broad in scope, the chapter places a particular 
        emphasis on the theory and methods used in the three main studies of 
        the thesis.

    \item in \autoref{chap:survival-analysis}, I give a
        overview of survival analysis, which is an essential field
        of statistics utilised throughout all three studies.
        In addition, it introduces the relevant theory
        and approaches for modeling time-to-event data
        with neural network models. 
        Specifically, it introduces the 
        discrete-time logistic-hazard approach used in \studyii{},
        which we further extended upon in \studyiii{}.

    \item The final background chapter,
        \autoref{chap:data-foundation}, offers a general description of 
        the different databases and registries used in the studies,
        providing crucial context for understanding the data foundation 
        of our research.

    \item In \cref{chap:study1-outline,chap:study2-outline,chap:study3-outline},
        I summarise each of the three included manuscripts.
        The summaries outline important methodological details,
        highlight the main research findings, and present the main
        conclusions from these.

    \item In \autoref{chap:findings-and-limitations},
        I summarise the principal findings of the thesis project,
        give an overview of both general and study-specific limitations,
        and conclude with some perspectives for future research.

    \item The appendices:
        \cref{chap:study1-paper}, 
        \cref{chap:study2-paper}, and
        \cref{chap:study3-paper} 
        includes the three full-length scientific
        manuscripts that form the core of this thesis.
\end{itemize}

