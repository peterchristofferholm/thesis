\begin{@empty}
    
\chapter{Summary of Thesis}
\setlength{\parskip}{6pt}

In modern medicine, 
ever increasing amounts of data 
is continuously being generated and recorded. 
Electronic health records,
although not maintained for research purposes,
stands as a unique and valuable source of real-world data
with the potential to revolutionise precision medicine.
This thesis explores the role of \ac{ML} in extracting 
clinically meaningful insights from such large-scale heterogenous health data.
With a primary focus on \ac{IHD},
a global leading cause of morbidity and mortality, 
the thesis presents three original studies 
under the framework of \ac{ML}-based precision medicine.

In \studyi{}, 
we used unsupervised clustering analysis to explore the comorbidity landscape
of \num{72249} patients with \ac{IHD}.
Using the time of the first \acl{CAG} or \acl{CCTA} as the index date,
we defined multimorbidity from the entire spectrum of diagnosis codes
in prior hospital records.
By constructing a patient similarity network from this data, 
we applied the Markov cluster algorithm,
a scalable graph-based clustering method, 
to identify \num{31} distinct clusters 
characterized by distinct patterns of multimorbidity 
and specific risks of subsequent outcomes.
The study's findings can be used to identify knowledge gaps that exist for 
patient subgroups with specific patterns of multimorbidity, 
which are often excluded from clinical trials.

In \studyii{}, 
we present the development and validation of 
a neural network-based survival model
for prediction of all-cause mortality in patients with \ac{IHD}.
This model, \pmhnet{1}, was 
developed using a large and diverse dataset of 
\num{39746} \ac{IHD} patients from the \ac{EDHR}
and incorporates a comprehensive set of 584 features,
including diagnosis history, procedural codes, laboratory test results,
and clinical measurements.
The model's performance was assessed using \ac{tdAUC} and the Brier score, 
and was compared against the \acs{GRACE} risk score 2.0. 
In the test set, \pmhnet{1} demonstrated a \ac{tdAUC} of 0.88 at both six months and one
year, 0.84 at three years, and 0.82 at five years, showing a notably higher
performance than both GRACE2.0 and other simpler models.
External validation on an independent Icelandic dataset of \num{8287} patients 
further showed that the model performance is generalizable.
This study establishes \pmhnet{1} as a valuable tool for assessing
risk of all-cause mortality in a real-world cohort of \ac{IHD} patients,
and can potentially aid clinicians in making informed decisions 
about patient care and interventions.

In \studyiii{}, 
we introduce \pmhnet{2}, an advanced iteration of our 
\ac{IHD} prognostication algorithm.
This updated version predicts three new outcomes,
which includes cause-specific mortality,
new iscemic events, and the development of 
\ac{IHD} complications, including heart failure and 
cardiac arrest.
The study's key contributions are twofold: 
firstly, it presents a novel framework for neural network-based discrete-time 
models capable of modelling time-to-event data with competing risks. 
Secondly, it offers an updated version
of our \acsfont{AI}-driven prognostication tool, 
equipped to predict a broader range of disease-relevant outcomes 
beyond all-cause mortality. 
Our competing-risks framework,
which can be viewed as an extension to discrete-time approach by 
Gensheimer and Narasimhan (2019), 
has been developed into the open-source Python package \textsf{DiscoTime}
(available through PyPI or Github). 
The included manuscript, while still a work in progress, 
effectively showcases the potential and capabilities 
of our proposed methodology and refined models.

\begin{otherlanguage}{danish}
\chapter*{Dansk Resumé}

I moderne medicin genereres og registreres der løbende store mængder data.
Elektroniske patientjournaler 
er ikke oprindeligt designet til at understøtte forskning,
men udgør ikke desto mindre en unik og særdeles værdifuld kilde til
realtidsdata der potentiel kan være med til at revolutionere præcisionsmedicin.
I denne PhD-afhandling undersøger jeg hvordan maskinlæring 
(eng: machine learning) kan anvendes til at 
udtrække klinisk relevant indsigt og mening
fra store databaser indeholdende heterogene sundhedsdata. 
Med hovedfokus på iskæmisk hjertesygdom (IHD),
en globalt ledende årsag til morbiditet og mortalitet, 
præsenterer afhandlingen tre originale studierne
inden for rammerne af maskinlæringsbaseret præcisionsmedicin.

I Studie 1 anvendte vi uovervåget klyngeanalyse til at udforske 
sammensætningen af multimorbiditet hos \num{72249} patienter med IHD. 
Ved at bruge tidspunktet for patienternes første koronarangiografi eller 
koronar-CT som indexdato, 
definerede vi multimorbiditet fra hele spektret af tidligere 
registrerede diagnosekoder fra de patienternes hospitalsjournaler. 
På baggrund af disse data, konstruerede vi et patientsimilaritetsnetværk 
og anvendte derefter Markov-klyngealgoritmen, 
en skalerbar klyngemetode til grafstrukturer, 
til at identificere 31 distinkte patientundergrupper. 
Disse undergrupper var karakteriseret ved at have forskellige 
multimorbiditetsmønstre og tilhørende sygdomsrisici.
Studiets resultater kan bruges til at kortlægge klinisk relevante 
multimorbiditetsmønstre, hvilket kan bruges til at identificere 
patientundergrupper, der lider under multimorbiditet,
og som oftest udelades fra kliniske forsøg og derved reelt 
set ikke dækkes af eksisterende kliniske retningslinjer.


I Studie 2 præsenterer vi udviklingen og valideringen af en dybt neural
netværksbaseret overlevelsesmodel til forudsigelse af alleårsmortalitet hos
patienter med IHD. Denne model, PMHNet-1, blev udviklet ved hjælp af et stort
og mangfoldigt dataset af 39.746 IHD-patienter fra EDHR og inkorporerer et
omfattende sæt af 584 funktioner, herunder diagnosehistorie, procedurekoder,
laboratorieresultater og kliniske målinger. Modellens præstation blev vurderet
ved hjælp af tdAUC og Brier-scoren og sammenlignet med GRACE risikoscore 2.0. I
testsæt demonstrerede PMHNet-1 en tdAUC på 0,88 ved både 6 måneder og et år,
0,84 ved tre år og 0,82 ved fem år, hvilket viste en betydeligt højere
præstation end både GRACE2.0 og andre enklere modeller. Ekstern validering på
et uafhængigt islandsk dataset af 8287 patienter viste yderligere, at
modelpræstationen er generaliserbar. Denne undersøgelse fastslår PMHNet-1 som
et værdifuldt værktøj til at vurdere risikoen for alleårsmortalitet i en
realistisk kohorte af IHD-patienter og kan potentielt hjælpe klinikere med at
træffe informerede beslutninger om patientpleje og interventioner.

I Studie 3 introducerer vi PMHNet-2, en avanceret iteration af vores
IHD-prognostik algoritme. Denne opdaterede version forudsiger tre nye
resultater, herunder årsagspecifik mortalitet, nye iskæmiske hændelser og
udviklingen af IHD-komplikationer, herunder hjertesvigt og hjertestop. Studiets
hovedbidrag er tofoldigt: For det første præsenterer det en ny ramme for dybt
neural netværksbaserede diskrete tid modeller, der er i stand til at modellere
tid-til-begivenhedsdata med konkurrerende risici. For det andet tilbyder det en
opdateret version af vores AI-drevne prognosticeringsværktøj, der er udstyret
til at forudsige et bredere udvalg af sygdomsrelaterede resultater ud over
alleårsmortalitet. Vores konkurrerende risikoramme, som kan betragtes som en
udvidelse af diskret-tidsmetoden af Gensheimer og Narasimhan (2019), er blevet
udviklet til den åbne kildekode Python-pakke DiscoTime (tilgængelig via PyPI
eller Github). Det inkluderede manuskript, selvom det stadig er et arbejde i
gang, viser effektivt potentialet og mulighederne for vores foreslåede metode
og raffinerede modeller.
    
\end{otherlanguage}
\end{@empty}
