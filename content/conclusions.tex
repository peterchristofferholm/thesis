\chapter{Limitations}

A number of methodological limitations and perspectives is 
relevant to discuss in the context of \studyi{}.

A common limitation of clustering applications is 
the lack of established techniques for \enquote{validation}.
For supervised learning, its relatively straightforward to 
apply ones model to a test-set, ideally from a totally different population,
and then directly measure if the model is able to generalize.
The lack of defined labels in unsupervised clustering applications, however,
makes such a strategy impossible.
Instead, one can of course apply the clustering approach on a comparable 
cohort and then observe if the overall patterns detected are similar.
However, this is not really quantifiable and thus also represents
an unfeasible approach.
Other strategies involves ...
In the light of the presented study,
we argue that the results should be used for hypothesis-generation,
and instead one can try to validate specific patterns hightlighted by our research.
We do not claim that the presented clustering is necessarily the "best",
as described in <reference to earlier in the thesis>, 
such a clustering is highly context dependent.
In this work, we used clustering as a tool to condense the occurence of more
than 3000 different diagnosis codes into 31 clusters that was possible to 
both visualise and understand.


Acknowledging the inherent limitations of unsupervised clustering
methodologies, particularly the absence of established validation techniques,
we emphasize the need to approach the findings of our study with a
hypothesis-driven perspective. 


Unlike supervised learning, where models can be
evaluated on independent test sets, clustering algorithms lack defined labels,
making direct validation challenging. Nevertheless, utilizing clustering as a
tool to condense a vast array of diagnostic codes (over 3,000) into 31
interpretable clusters provides valuable insights into the multimorbidity
landscape associated with \ac{IHD}.


The methodological landscape of unsupervised clustering algorithms, such as
those employed in \studyi{}, is marked by a unique set of challenges and
considerations. One of the most prominent limitations is the absence of
established validation techniques. Unlike supervised learning, where models are
trained to classify or predict predefined labels, unsupervised clustering
algorithms operate in the absence of such labels. This inherent characteristic
makes the conventional approach of validating models on independent test sets
infeasible.

The lack of direct validation methodologies necessitates a shift in perspective
when evaluating clustering results. Rather than aiming for absolute validation,
researchers should adopt a hypothesis-driven approach, utilizing the insights
derived from clustering to generate testable hypotheses. This approach
acknowledges the context-dependent nature of clustering, where the validity of
the identified clusters is contingent on the specific dataset and research
objectives.

In the context of \studyi{}, the clustering methodology was employed to
condense a vast array of diagnostic codes (over 3,000) into 31 interpretable
clusters. These clusters represent distinct patterns of co-occurring
comorbidities associated with \ac{IHD}. While direct validation of these
clusters is challenging, the patterns they reveal can serve as valuable
starting points for further investigation and hypothesis generation.

It is important to emphasize that the presented clustering approach is not
intended to be the definitive or universally applicable solution for exploring
the complex multimorbidity landscape of \ac{IHD}. Instead, it serves as a
valuable tool for generating hypotheses and guiding future research. As
research progresses and understanding of the underlying mechanisms of
multimorbidity deepens, more sophisticated clustering techniques and validation
strategies may emerge.

In summary, while the lack of established validation methods poses a challenge
for unsupervised clustering, adopting a hypothesis-driven approach and
acknowledging the context-dependent nature of clustering can still provide
valuable insights into complex data structures. The clustering results obtained
in \studyi{} have generated promising hypotheses that warrant further
investigation and validation. As the field of multimorbidity research continues
to advance, the methodological landscape of clustering is likely to evolve as
well, enabling more robust and generalizable analyses of complex comorbidity
patterns.


\chapter{Conclusions and Future Perspectives}
\label{chap:conclusions}

\section{Study I}

We primarily used hospital data which might result in underrepresentation of
conditions commonly managed in primary care settings,
including non-complex infections, hypertension, and soft-tissue disorders.
~\autocite{finleyWhat2018}

Leveraging population-level genomic datasets from large biobanks,
it should be possible to implement genetics informed risk stratification
for routine clinical use.


\section{Topics to be discussed}
\begin{itemize}
    \item Consider doing ablation studies to produce less-complex versions
        of the predictions models.
    \item Lack of prospective validation of medical prediction models
    \item Lack of clinical deployment and adoption 
    \item Limitations of explainable AI
\end{itemize}


How important is XAI? 
There have been alot of discussion about, and there are many opinions on, 
the black-box nature of neural networks and 
how it should or should not affect the clinical use of such models.
\autocite{gunningXAI2019, vanderveldenExplainable2022}

\textquote[russellArtificial2009]{%
    Which would you trust: 
    an experimental aircraft that has never flown before
    but has a detailed explanation of why it is safe, 
    or an aircraft that safely completed 100 previous flights 
    and has been carefully maintained,
    but comes with no guaranted explanation?
}

\textquote[knuuti20192020]{% class I recommendation
    Invasive coronary angiography is recommended as an alternative test to
    diagnose \textsc{CAD} \textins{coronary artery disease} in patients with a
    high clinical likelihood, severe symptoms refractory to medical therapy or
    typical angina at a low level of exercise, and clinical evaluation that
    indicates high event risk.%
}

\textquote[knuuti20192020]{% class IIa recommendation
    Invasive coronary angiography with the availability of invasive functional
    evaluation should be considered for confirmation of the diagnosis of CAD in
    patients with an uncertain diagnosis on non-invasive testing%
}

Evaluate the clinical impact and practical applicability of the developed ML models (PMHnetV1 and PMHnetV2) in real-world healthcare settings. This would involve assessing how these models influence clinical decision-making, patient outcomes, and healthcare resource allocation for patients with ischemic heart disease.
