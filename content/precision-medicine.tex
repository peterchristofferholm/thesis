\chapter{Data-driven Precision Medicine} \label{precision-medicine}

A 2013 Cochrane review on \citetitle{taylorStatins2013} concluded 
that statins effectively lower all-cause mortality and reduce the incidence
of both fatal and non-fatal cardiovascular events all without any serious
adverse effects~\autocite{taylorStatins2013}.
For instance, the relative risk of fatal cardiovascular events 
when using statins as opposed to placebo
was estimated to \num{0.82} with a 
\si{95}{\%} \ac{CI} of \num{0.70} to \num{0.96}.
~\autocite{taylorStatins2013}
This means that those treated with statins, on average,
are \si{18}{\%} less likely to die from cardiovascular causes. 
However, it is important to emphasize that this \si{18}{\%} 
reduction is an average effect.
There might be groups of people for whom the reduction could be 
even higher, while others may see little to no benefit. 
Understanding and making use of such individual variability 
is the central objective of precision medicine.

As a concept, precision medicine is not a new idea;
tissue- and blood typing, for instance,  
has been used to guide treatment for many decades.
However, the prospective of leveraging large clinical databases
and broad array of phenotypic information 
to deliver precision medicine across
a wide spectrum of disease states
is adding renewed interest in the concept.






Use data to improve decision making


Subgroup identification seeks to identify groups of individuals with an
increased treatment response, which
\citeauthor{kosorokPrecision2019} refers to as 
\textquote[kosorokPrecision2019]{%
    finding the right patient for the right treatment%
}.


\begin{table*}[h]
    \footnotesize
    \centering
    \begin{tabularx}{0.9\textwidth}{XlX} \toprule
    Condition & Gene & Action \\
    \midrule
    Familial hypercholesterolaemia & \textit{PCSK9}, \textit{APOB}, and \textit{LDLR} & Indication for PCSK9 inhibitor drugs \\
    Familial hypercholesterolaemia & \textit{PCSK9}, \textit{APOB}, and \textit{LDLR} & Indication for PCSK9 inhibitor drugs \\
    Familial hypercholesterolaemia & \textit{PCSK9}, \textit{APOB}, and \textit{LDLR} & Indication for PCSK9 inhibitor drugs \\
    Familial hypercholesterolaemia & \textit{PCSK9}, \textit{APOB}, and \textit{LDLR} & Indication for PCSK9 inhibitor drugs \\
    Chronic myeloid leukemia       & BCR/ABL & Imatinib \\
    Chronic myeloid leukemia       & BCR/ABL & Imatinib \\
    Chronic myeloid leukemia       & BCR/ABL & Imatinib \\
    Chronic myeloid leukemia       & BCR/ABL & Imatinib \\
    Chronic myeloid leukemia       & BCR/ABL & Imatinib \\
    \bottomrule
    \end{tabularx}
    \caption[Precision drugs]{
        Examples of precision pharmacotherapy informed by genetics 
        and more more more more
    }
\end{table*}



Electronical health records are a rich source of health data,
and can be used to find connections between risk factors and diseases.

Precision medicine is prevention and treatment approaches
that takes individual variation into account.


The terms
\enquote{precision},
\enquote{personalized},
and \enquote{individualized medicine}
are often used interchangeably.






\section{Deep phenotyping} \label{deep-pheno}

In their review 
\citetitle{konigWhat2017}, 
\citeauthor{konigWhat2017} presents data-driven precision medicine
as a framework with three main \enquote{tracks} or \enquote{processes}
that serves as an useful abstraction for understanding
the process of moving from big data
to clinically actionable insights~%
\autocite{konigWhat2017}. 
At the center of the framwork resides the data,
which are both plentiful and representative of the population of interest.
Now, the three tracks are
1) preprocessing and data-mining
2) diagnostic and prognostic models
and 3) prediction of treatment response.

\subsection{Track 1: preprocessing and data mining}

Quality control, preprocessing, and extraction of information.
The concrete steps and methods utilized as part of this track
is highly context-dependent, 
and is therefore difficult to describe on a general level. 

Feature-engineering.

Extracting structured information from unstructured information.

Clustering.

\subsection{Track 2: diagnostic and prognostic models}

Relying in part on processed data from track 1
as well as any insights gained from data-mining,
track 2 is concerned with the development
of models that can characterise current (diagnostic models)
or future (prognostic models) states of health and disease~%
\autocite{konigWhat2017}. 

\subsection{Track 3: predicting treatment response}

\todo[inline, caption={Some text}]{
\begin{minipage}{\linewidth}
    \emph{Notes:}

    Causality.
    Different strategies exists for guiding the development of 
    predicting therapy response.
    We might be able to use prognostic and diagnostic factors
    discovered in the previous track to predict treatment response.
    Example: HER is prognostic for survival in breast cancer,
        but is also predictive of treatment response against X.


\end{minipage}
}

As with the other tracks, track 3 gan generate further knowledge 
that in turn can feed back to the more general phenotyping of patients
\autocite{konigWhat2017}. 


\section{Health Informatics}

Health informatics is a cross-disciplinary field
that aims at developing methods and techniques
for the collection, study, and analysis of healthcare data.
It operates in the intersection between medicine and computer science,
and involves disciplines such as
bioinformatics, software engineering, statistics, information systems,
data science, and artificial intelligence.
The overall goal of health informatics is the data-driven improvement
of delivery and accuracy of health and healthcare for the individual patient.

In modern medicine, 
ever increasing amounts of data 
is continuously being generated and collected.
Ranging from structured administrative data 
used primarily for billing purposes
to advanced imaging and high-througput \enquote{omics} analyses,
the array of available data is as diverse as it is plentiful.

The volume and range of information collected
at just a single visit for a single patient
can be large.


There is increasing evidence that informatics improves
health care, public health, and biomedical research.

    
An electronic health record
is a systematized collection of health data 
stored in a digital format.
Electronic health records may include data such as
medical history, drug prescriptions, laboratory test results,
radiology images, body measurements, and billing information.
EHRs enable following patients in both time and space;
not only in the physical space that is the cardiology department
and the intensive care unit, but also in the health-and-disease space.

\begin{itemize}
    \item \textbf{Decision support systems}
    \item \textbf{Personalized medicine}
    \item \textbf{Mobile health applications} (aka. \textit{mHealth})
    \item \textbf{Screening}
    \item \textbf{Surveillance}
    \item \textbf{Medical outcomes}
\end{itemize}

\marginnote{
    The five Vs of big data:
    \begin{itemize}
        \item Volume
        \item Velocity
        \item Variety
        \item Value
        \item Veracity
    \end{itemize}
}

\section{Clinical decision support systems}

Deep learning can identify patterns in sparse, noisy,
and very heterogenous data without the need for expert feature engineering
\cite{norgeotCall2019}.

Challenges in developing machine learning models from electronic health records
includes syntactic interoperability, 
which specifies the format and structure of the data.
Many different interopability formats exists, 
but their adoption varies considerably.

Benefits of interopable digital health data
~\autocite{lehneWhy2019}.
Test~\cite{lehneWhy2019}.

\begin{itemize}
    \item Large-scale observational studies.
    \item Less time spend on data cleaning and pre-processing.
    \item Interoperable exchange format could further cross-institutional
    and international collaboration, which would make it easier 
    to reproduce and externally validate e.g. prognostic applications.
    In the case of rare diseases with very limited number of patients, 
    international pooling of data could enable analysis
    that otherwise would not be possible.
    \item Faster research and development process.
    \item If data is known to conform to a well-defined format,
    computer software can be written without explicit access to the data.
    This solve many issues related to sensitive data or
    data that are otherwise subject to strict data protection regulations.
\end{itemize}


Healthcare data are usually private and scattered across various applications,
which makes it difficult to share data and generate robust results 
that translate to different and diverse populations.


Succesful implementation of data-driven applications
require large and diverse data sets.

AI approaches rely on data that accurately represent
the real-life distribution of the underlying problem.
We can not exclusively rely on data that have been carefully curated 
from few and often very similar sources. 
In order to capture subtle relationships 
between health and disease patterns,
we need to include many and diverse cases~%
\autocite{riekeFuture2020}.

Federated learning could solve some of these issues\autocite{riekeFuture2020}.




