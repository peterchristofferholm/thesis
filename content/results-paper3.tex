\chapter{Study III: Time-to-Event Prediction with Competing Risks}
\label{outline-study-3}

\section{Model formulation}

The proposed model parametrizes the cause-specific conditional hazards,
and the output \(\Lambda\) is then a 
\(n \times q \times (m + 1)\) matrix of probabilities
where values across the last dimension all sum up to 1.


In the case of several competing target events,
it is necessary to define several hazard functions,
one for each distinct outcome type.
Let \(R \in \{1, ..., m\}\) denote the competing risks.
Then, for disrete time \(T \in \{1, ..., q\}\), 
the cause-specific hazard function from cause $r$ is defined by
%
\begin{equation*}
    \lambda_{r}(t | \mathbf{x}) = \Pr (T = t, R = r | T \geq t, \mathbf{x})
\end{equation*}
%
where \(\mathbf{x}\) is a vector of time-independent predictor variables. 
The different hazard functions, 
\(\lambda_{1}(t|\mathbf{x}), ..., \lambda_{m}(t|\mathbf{x})\) 
can be combined into the overall hazard function defined as
%
\begin{equation}
    \lambda(t|\mathbf{x}) 
    = \sum_{r=1}^{m}(\lambda_{r}(t|\mathbf{x}))
    = \Pr(T = t | t \leq T, \mathbf{x})
\end{equation}
%
From this it follows that the survival function is
%
\begin{equation}
    S(t|\mathbf{x})
    = \Pr(T >= t|\mathbf{x})
    = \prod_{i=1}^{t}(1 - \lambda(i|\mathbf{x}))
\end{equation}

The model uses $m + 1$ categorical responses.
The responses are either any of the $m$ competings risks, 
or conditional survival.
%
\begin{equation}
    \Pr(T > t | T \geq t, \mathbf{x})
    = 1 - \sum_{r = 1}^{m} \lambda_r (t | \mathbf{x})
    = 1 - \lambda (t | \mathbf{x})
\end{equation}

\section{Model formulation}

Continuous follow-up time is divided into into $q$ intervals
%
\begin{equation*}
	[0, a_1), [a_1, a_2), ..., [a_{q-1}, a_q), [a_{q}, \infty)
\end{equation*}
%
and the time to event variable \(T\) is denoted 
with a positive integer \(t = 1, 2, ...\),
which points to the interval \([a_{t-1}, a_{t})\).
With this discrete time scale,
the cause specific hazard function \(\lambda_{r}(t)\) is a conditional probability defined as
%
\begin{equation*}
    \lambda_{r}(t | \mathbf{x}) = \Pr (T = t, R = r | T \geq t, \mathbf{x})
\end{equation*}
%
given a vector \(\mathbf{x}\) of predictor variables.

The discrete hazard function is defined as
the probability of failure in the interval \(t\),
given that the individual is still alive at the end of the preceding interval
\(t - 1\).

In a time-to-event model with several competing target events,
we describe the process with multiple hazard functions, 
that each describe a specific target outcome or risk.
Typically, 
we refer to the individual hazard functions as cause-specific hazards.

For discrete time \(T \in \{1, ..., s+1\}\), and with a given vector \(\mathbf{x}\) of time-independent covariates,
the cause-specific hazard from cause \(r\) is defined as

\begin{equation}
    \lambda_{r}(t|\mathbf{x}) 
    = P(T = t, R = r | t \leq T, \mathbf{x})
\end{equation}

Which represents the probability of 

The overall hazard function is

\begin{equation}
    \lambda(t|\mathbf{x}) 
    = \sum_{r=1}^{m}(\lambda_{r}(t|\mathbf{x}))
    = P(T = t | t \leq T, \mathbf{x})
\end{equation}

The survival function is the unconditional probability of an event in period \(t\) given the specific covariates is 

\begin{equation}
    S(t|\mathbf{x})
    = P(T >= t|\mathbf{x})
    = \prod_{i=1}^{t}(1 - \lambda(i|\mathbf{x}))
\end{equation}

From the hazard we can obtain the survival function 
%
\begin{equation*}
    S(t | \mathbf{x}) 
    = \Pr (T \geq t | \mathbf{x}) 
    = \prod_{s = 1}^{t-1} (1 - \lambda(s | \mathbf{x}))
\end{equation*}
%
which represents the unconditional probability of surviving interval \(t\) given the specific covariates cf. \autocite{tutzModeling2016}.

Of note, this interpretation is considerably more accessible than the usual
continuous hazard one.


\begin{equation}
    S(\tau) = P(T > \tau) = \sum_{i = 1}^{\tau - 1} (P(T = \tau - i))    
\end{equation}

The cause specific cumulative incidence function is

\begin{equation}
	CIF_r(t|\mathbf{x})
	= \sum_{i=1}^{t}\lambda_{r}(t|\mathbf{x}) S(t-1|\mathbf{x})
\end{equation}

Let the data be given by \(t_{i}, r_{i}, \delta_{i}, \mathbf{x}_{i}\),
where \(r_{i} \in \{1, ..., m\}\) indicates the target event. 
Assuming random censoring at the end of the interval
with \(t_i = \min\{T_i, C_i\}\), 
where events are defined by the indicator function

\begin{equation}
\delta_i = 
	\begin{cases}
		1, & T_i \leq C_i \\
		0, & T_i > C_i
	\end{cases}
\end{equation}

\begin{subequations}
\begin{equation}
	nll =
	- \sum_{i=1}^{n}
	\sum_{t=1}^{ti}
	\left(
		\sum_{r \in C} \delta_{itr} \log(\lambda_{r} (t | \mathbf{x}))
		+ \delta_{it0} \log\left(
			1 - \sum_{r \in C} \lambda_{r} (t | \mathbf{x})
		\right)
	\right)
\end{equation}

Or alternatively 

\begin{equation}
	nll =
	- \sum_{t=1}^{q}
	\sum_{i \in R_s}
	\delta_{is} \log(s|\mathbf{x}_i) 
	+ (1 - \delta_{is}) (1 - \log(s|\mathbf{x}_i)))
\end{equation}
\end{subequations}

In our approach, the neural network parametrises the logistic hazards.
The goal is to learn \( \hat{h}(t | \mathbf{x}) \)
for each of the competing events.




The discrete probability function

\begin{equation}
    f_l = \Pr (T \in A_l) = S(t_{l-1}) - S(t|l)
\end{equation}

the discrete hazard rate

\begin{equation}
    h_l = \Pr (T \in A_l | T > t_{l-1}) = \frac{f_l}{S(t_{l-1})}
\end{equation}

    
The contribution to the likelihood function of the \textit{i}th subject

For uncensored subjects, the contribution is
%
\begin{equation}
    \Pr (T_i \in A_{l_i}) = f_{l_i} = h_{l_i} \prod_{l=1}^{l_i-1}(1-h_{l_i})
\end{equation}
%
while for censored subject, the contribution is
%
\begin{equation}
    \Pr (T_i > t_{l_i}) = S(t_{l_i}) = \prod_{l=1}^{l}(1-h_{l_i})
\end{equation}

A similar approach have previously been described \autocite{biganzoliFeed1998}.

