\chapter{Time-to-Event Prediction with Neural Networks}
\label{survival-analysis}

% marginnote {{{
\marginnote{%
    \setlength{\parindent}{0pt}
    \vskip 1em
    What is survival analysis?
    \begin{description}[leftmargin=!, labelwidth=3em]
        \item[outcome] time until an event occurs.
            Can be measured in seconds, days, months, etc.
        \item[event] death, relapse, remission, engine failure, etc.
    \end{description}
    
    \begin{center}
    \begin{tikzpicture}
        \noindent
        \node (a) at (2.5, 0) {event};
        \node (b) at (0, 0) {start (\(t_0\))} edge ["time", ->] (a);
    \end{tikzpicture}
    \end{center}
}
% }}}

In the previous chapter, 
I gave an overview of machine learning and neural networks,
highlighting key ideas and concepts related to the studies in this thesis.
Neural networks specifically, 
were used in \studyii{} and \studyiii{} to develop
prediction models for ischemic heart disease.
These models, however, diverge from classical neural network methods.
Instead, they include modifications that make them applicable
in modelling and analysis of time-to-event data.
This chapter will provide an introduction to the fundamentals
of survival analysis and will cover the
theory that enables the implementation of such analyses with
neural network models.
% The chapter concludes with a discussion on different 
% validation methods for time-to-event prediction models.

\section{What is Survival Analysis?}

Generally, 
survival analysis is the collection of statistical methods
for the modelling and analysis of time-to-event data,
which is data where the outcome variable of interest 
is the time until \enquote{something} happens.~%
~\autocite{kleinbaumSurvival2011}
This \enquote{something} is a particular event of interest,
which, depending on the type of analytical problem, 
could be cancer relapse, 
diabetes remission,
or death.
In cardiovascular research, 
common examples of time-to-event outcomes include
\begin{enumerate*}
    \item time to death due to any cause (all-cause mortality)
    \item time to death due to a specific cause,
        e.g. sudden cardiac arrest
    \item time to first occurence of a \ac{MACE}
\end{enumerate*}.
To figure out what processes and characteristics 
that are associated with such events, 
in survival analysis, we try to model the relationship between
explanatory variables and the number of weeks, months, or years 
until that particular event is likely to occur. 

% marginnote{{{
\marginnote{%
    Survival analysis have applications outside biomedical research.
    In engineering, it is called \textit{reliability analysis} and
    is used to model the time-to-failure of system-critical components 
    such as e.g. bearings or valves.
}% }}}

Although this task can be daunting in its own right, 
an additional complication to survival analysis 
is the presence of observations that are subject to 
censoring.
This concept, censoring, refers to cases 
where the event of interest has not been observed 
before the end of follow-up, 
e.g. when a study or experiment has to be stopped.
In such cases, 
we would know that a given subject did not experience a relapse 
in the three months he or she was included in the study, 
but after the study period ends, 
we have no information on the status of the patient. 
Including and utilizing this partial information
is a cornerstone in many survival analysis problems.

There exists different forms of censoring,
such as right censoring, left censoring, and interval censoring.
In the study designs used throughout this thesis 
we have only had to deal with right censoring,
the most common form of censoring,
so the two other types will not be described further.
See instead the text book by \citeauthor{kleinSurvival2003} 
for more details on this.
~\autocite{kleinSurvival2003}

\section{Fundamentals of Survival Analysis}

In survival analysis, 
the central outcome variable is survival time,
a non-negative random variable denoted as \(T\). 
When refering to specific values of \(T\), 
a lower case \(t\) is typically used.  
A survival dataset \(\mathfrak{D}\) of size \(N\) is given by
\begin{equation}
    \mathfrak{D}_N = \{(t_i, \sigma_i, \vec{x}_i) \mid i = 1, \ldots, N\} 
\end{equation}
where \(t_i = \min(T_{i}, C_i) \) is the survival time 
for the \(i\)th subject,
with \(T_i\) denoting the survival time
and \(C_i\) denoting the censoring time. 
Also, \(\vec{x}_i = (x_1, x_2, \dots, x_p)'\) is the covariate vector
and \(\sigma_i\) is the event indicator, which is defined as
\begin{equation}
    \label{eq:sigma-def}
    \sigma_i =
        \begin{cases}
            0 & \text{if subject is censored} \; (T_i >    C_i) \\
            1 & \text{if event is observed} \; (T_i \leq C_i)
        \end{cases}
\end{equation}

In the following, I will initially be assuming that \(T\) is 
continuous and that there is an absence of competing risks, 
however both of these assumptions will later be relaxed in the discussion 
of competing risks and discrete-time survival analysis.

\subsection{Basic Survival Quantities}
\label{sub:survival-quantities}

% figure: theoretical survival function{{{
\begin{marginfigure}%
	\begin{tikzpicture}[scale=2]
	  \draw[->] (0, 0) --  (2,0) 
		node[pos=0.0, below] {$0$}
		node[pos=0.5, below] {$t$}
		node[pos=1.0, below] {$\infty$};
	  \draw[->] (0, 0) --  (0,1) 
		node[pos=1.0, left] {$1$}
		node[pos=0.5, left] {$S(t)$};
	  \draw[-, color=color2, thick] 
		(0.05, 0.95) .. controls (1, 1) and (1, 0) .. (1.90, 0.05);
	\end{tikzpicture}
    \caption[A Theoretical Survival Function]}}

In survival analysis, 
the central function of interest is 
the survival function \(S(t)\), 
that represents the probability 
of an individual still being alive after 
some specified duration of time, we have that 
%
\begin{equation}
    S(t) = \PR (T > t), \quad 0 < t < \infty.
\end{equation}

The survival function is 
the integral of the probability density function, \(f(t)\),
and is the complement to the cumulative distribution function, \(F(t)\),
which means that
~\autocite{kleinSurvival2003}
%
\begin{equation}
    S(t) = 1 - F(t) 
    \quad \text{and} \quad 
    S(t) = \int_{t}^{\infty} f(u) \, \diff u
\end{equation}

Another fundamental quantity is the hazard function, or hazard rate,
which represents the instantaneous failure rate at a given timepoint,
and is defined as
%
\begin{equation}
    \label{eq:hazard-function}
    \lambda(t) = \lim_{\Delta t \to 0} 
        \frac{\PR (t \leq T < t + \Delta t \mid T \geq t)}{\Delta t}
\end{equation}
% 
from which it can be shown that
~\autocite{kleinSurvival2003}
%
\begin{equation}
    \lambda(t) = \frac{f(t)}{S(t)} = -\frac{\diff}{\diff t} \ln[S(t)].
\end{equation}
%
and thus the hazard function completely describes the distribution of \(T\),
such that all the the other quantities can be obtained from it---%
as well as the other way around.

In terms of its interpretation, 
from \cref{eq:hazard-function} it follows that \(\lambda(t)\Delta t\) 
is a measure of the conditional probability of failure in a small time
window, given that the individual is still alive at time \(t\).
~\autocite{kleinSurvival2003}

Analogous to the relation between \(f(t)\) and \(F(t)\), 
integrating \(\lambda(t)\) with resport to \(t\),
we obtain cumulative hazard function, defined as
%
\begin{equation}
    \Lambda(t) = \int_{0}^{t} \lambda(u) \, \diff u = -\ln[S(t)].
\end{equation}

\subsection{The Kaplan-Meier Estimator}

The survival function of a population,
can be estimated using the Kaplan-Meier method,
which is the standard estimator of the survival function.
~\autocite{kaplan1958nonparametric}
~\autocite{kleinSurvival2003}
In this approach, 
the distinct failure times are ordered such that%
\footnotemark
\begin{equation*}
    t_{(1)} < t_{(2)} < \ldots < t_{(j)},
\end{equation*}
%
and we introduce two quantities to keep track of
the number of failures \(\widebar{D}(j)\), 
as well as the number of subjects still at risk \(\widebar{A}(j)\).
They are defined as
\begin{equation}
\begin{aligned}
    \bar{D}(j) &= \card \{i \in \{1, \dots, n\} \mid t_i = t_{(j)}, \sigma_i = 1\} \\
    \bar{A}(j) &= \card \{i \in \{1, \dots, n\} \mid t_i > t_{(j)}\},
\end{aligned}
\end{equation}
and the Kaplan-Meier estimator can then be formulated as 
\begin{equation}
    \widehat{S}(t)
    =   \prod_{j \mid t_{(j)} \leq t} 
        \frac{
            \bar{A}(j) -
            \bar{D}(j)
        }{
            \bar{A}(j)
        }
    =   \prod_{j \mid t_{(j)} \leq t} 
        1 - \frac{
            \bar{D}(j)
        }{
            \bar{A}(j)
        }.
\end{equation}

\footnotetext{%
    Following the example of [\cite{kleinbaumSurvival2011}], 
    the \(t\)'s denoted with subscripts within parentheses \(t_{(j)}\)
    refers to the \(j\)th element of the ordered distinct failure times and
    are thus different from \(t_1, t_2, \ldots, t_i\) that refers to the 
    observed failure time of subject \(1\), \(2\), and \(i\)
}

The Kaplan-Meier estimator is thus as step-function that decreases
after each observed event.
While the Kaplan-Meier estimator is useful 
for summarising the survival of a population, 
it does not account for the effect of covariates.
Instead, another approach is needed for regression analyses.

\subsection{Cox's Proportional Hazards Model}

To describe and model the relationship between explanatory variables
and time-to-event phenomenons, a widely used statistical model is 
the Cox proportional hazards model. 
~\autocite{coxRegression1972}
This model seeks to model the hazard function over time \(t\),
of an individual with a covariate vector \(\vec{x} = (x_1, x_2, \dots)'\),
and assumes that it takes the form of
%
\begin{equation}
    \label{eq:cox}
    \widehat{\lambda} (t \,|\, \vec{x}) = \hzt \exp [g(\vec{x})],
\end{equation}
%
where \(\hzt\)
is an unspecified baseline hazard function,
and \(g(\vec{x})\) is some parametric function.
For this reason, the Cox model 
is referred to as a semi-parametric model.
In its classical formulation, 
this function is a linear combination of parameters 
\(\vec{\beta}\) and covariates \(\vec{x}\),
as given by

\begin{equation}
    g(\vec{x}) 
    = \vec{\beta}' \vec{x} 
    = \beta_1 x_1 + \beta_2 x_2 + \ldots + \beta_p x_p
\end{equation}

In estimation of the parameters \(\vec{\beta}\),
the baseline hazard \(\hzt\) is treated as a nuisance function
and the coefficients are estimated by
maximising a partial likelihood
in which \(\hzt\) has been abstracted away.
~\autocite{kalbfleischStatistical2002}

A central assumption in the Cox model, 
at least in the standard version with fixed covariates
(\(\vec{\beta}\) instead of \(\vec{\beta}(t)\)),
is that of proportional hazards.
Let \(\vec{x}\) and \(\vec{x}'\) be two different 
covariate vectors, now the ratio between their
respective Cox-estimated hazards is

\begin{equation}
    \label{eq:hazard-ratio}
    \begin{aligned}
    \frac%
        {\widehat{\lambda}(t \,|\, \vec{x} \hfill)}%
        {\widehat{\lambda}(t \,|\, \vec{x'})}
    &=
    \frac%
        {\hzt \exp (\vec{\beta}\cdot\vec{x}\hfill)}%
        {\hzt \exp (\vec{\beta}\cdot\vec{x'})} \\
    &=
    \frac%
        {\exp (\vec{\beta}\cdot\vec{x}\hfill)}%
        {\exp (\vec{\beta}\cdot\vec{x'})} \\
    &= \exp (\vec{\beta} \cdot (\vec{x} - \vec{x'}))
    \end{aligned}
\end{equation}

Since the right-hand side of the equation does not include a term for \(t\),
the hazard ratio between the two samples are constant and 
they are thus proportional to one another., 
This shows that by assuming the hazard takes the form of \cref{eq:cox},
then it is also assumed that the hazards between two subjects are proportional.
Although this assumption is a strong one, 
and the validity of the Cox model relies on it, 
the assumption makes interpretation of parameters easier.
~\autocite{tutzModeling2016}
For example, 
in an randomized clinical trial
studying the survival effect of a new type of medication, 
we can let \(x = 1\) represent the experimental treatment  
and \(x' = 0\) represent standard of care, 
then the hazard ratio in \cref{eq:hazard-ratio} takes the form of
%
\begin{equation}
      \exp \left(\beta (x - x')\right)
    = \exp \left(\beta (1 - 0)\right)
    = \exp (\beta ),
\end{equation}
%
which means that if \(\beta < 0\), 
then the hazard of the experimental treatment is 
\(\exp({\beta})\) times lower than standard of care
and should therefore be preferred.%
\sidenote{% 
This example is a slightly modified version of the 
one given in \cite[pp. 50]{tutzModeling2016}}


\section{Analysis of Competing Risks}

\begin{marginfigure}[3em]% {{{
    \tikzstyle{outcome}=[%
        rectangle, rounded corners, minimum height=5mm, fill=color3
    ]
    \centering
    \begin{tikzpicture}[x=0.60\linewidth, y=1cm]
    \graph [edge quotes={font=\scriptsize, fill=white}, 
            nodes      ={draw, outcome, sloped, minimum width=1cm}]{
        alive [fill=color4] -> dead [> "\(\lambda(t)\)" ];
    };
    \end{tikzpicture}
    \caption[A Single State Model]{
        A simple survival analysis setup 
        involves modelling a single transition between states 
        \enquote{alive} and \enquote{dead}.
    }
    \label{fig:ssm}
\end{marginfigure}% }}}

Up to this point, the description of concepts in survival analysis has
assumed the presence of only a single event type, such as all-cause mortality
(\cref{fig:ssm}).
In practice, particularly in clinical settings, 
this single-event model can be too restrictive,
and instead one needs to consider competing risks
(\cref{fig:msm}).
By definition, a competing risk is a secondary event whose occurence 
prevents the primary event from occuring.
For example,
in a study where the primary outcome is cardiovascular mortality,
deaths from non-cardiovascular causes are a competing risk.

In the following, 
we let \(R \in \{1, \dots, \kappa\}\) denote the \(\kappa\) different competing risks,
and then slightly update the definition of the event indicator \(\sigma_i\)
(\cref{eq:sigma-def}), such that
\begin{equation}
    \label{eq:sigma-def-2}
    \sigma_i =
        \begin{cases}
            0 & \text{if subject is censored} \; (T_i >    C_i) \\
            r & \text{if cause r is observed} \; (T_i \leq C_i, R = r).
        \end{cases}
\end{equation}

\begin{marginfigure}% {{{
    \tikzstyle{outcome}=[%
        rectangle, rounded corners, minimum height=5mm, fill=color3
    ]
    \centering
    \begin{tikzpicture}[x=0.60\linewidth, y=0.85cm]
    \graph [edge quotes={font=\scriptsize, fill=white}, 
            nodes      ={draw, outcome, sloped, minimum width=1cm}]{
        alive [fill=color4] -> {
            cause 1 [> "\(\lambda_1(t)\)" ],
            cause 2 [> "\(\lambda_2(t)\)" ],
            cause k [> "\(\lambda_\kappa(t)\)" ],
        };
    };
    \end{tikzpicture}
    \caption[A Multi-State Model]{
        A survival analysis setup with competing risks
        involves modelling transitions between states 
        \enquote{alive} and \(k\) different absorbing
        states, \enquote{cause 1} to \enquote{cause \(\kappa\)}
    }
    \label{fig:msm}
\end{marginfigure}
% }}}

\subsection{Cause-Specific Survival Quantities}

To describe time-to-event phenomena with competing risks, 
we introduce the cause-specific hazard function and 
cumulative-incidence function.
With \(R \in \{1, \dots, \kappa\}\) denoting the \(\kappa\) different competing risks, 
the cause-specific hazard function is defined as
\begin{equation}
    \lambda_r(t) = \lim_{\Delta t \to 0} 
        \frac{\PR (t \leq T < t + \Delta t, R=r \mid T \geq t)}{\Delta t}
\end{equation}
where \(r\) refers to a specific value of \(R\).
The cause-specific cumulative incidence function is defined as
~\autocite{kalbfleischStatistical2002}
\begin{equation}
    F_r(t) = \PR(T \leq t, R = r).
\end{equation}

The overall hazard and cumulative incidence, 
which combines failures of any of the \(\kappa\) causes,
correspond to the hazard function 
and the cumulative distribution function 
in the single-event setting, that is
\begin{equation}
    \lambda(t) = \sum_{r=1}^{\kappa} \lambda_r(t)
    \quad \text{and} \quad
    F(t) = \sum_{r=1}^{\kappa} F_r(t).
\end{equation}

\subsection{The Aalen-Johansen Estimator}

In the competing risk setting, 
some of the methodology previously presented
have to be slightly adjusted.
For example, 
simply treating competing events as censored 
and applying the standard Kaplan-Meier estimator, 
would lead to a biased estimate of \(F(t)\).
~\autocite{pepeKaplan1993}
Instead, an alternative approach is the Aalen-Johansen estimator
that allows estimation of the cause-specific cumulative incidence.
~\autocite{aalenEmpirical1978}
Of note, the Aalen-Johansen is a general method for estimating
transition probabilities in state-transition models,
and can be used to describe complex multi-state models,
including those with repeated events and with non-terminal states.
~\autocite{survival-package}
However, we will be assuming a standard competing-risk setting
with \(\kappa\) different terminal states, 
as depicted in \cref{fig:msm}.
 
If we again order the distinct failure times, 
corresponding to any cause, 
such that
\(t_{(1)} < t_{(2)} < \ldots < t_{(j)}\),
and update the definition of \(\bar{D}(j)\) to keep track of cause-specific
events, such that we have
\begin{equation}
\begin{aligned}
    \bar{D}(j, r) &= \card \{i \in \{1, \dots, n\} \mid t_i = t_{(j)}, r_i = r\} \\
    \bar{A}(j)    &= \card \{i \in \{1, \dots, n\} \mid t_i > t_{(j)}\}.
\end{aligned}
\end{equation}
Now, the Aalen-Johansen estimator of the cumulative incidence function
can be defined as 

\begin{equation}
    \widehat{F}_r(t)
    =   \sum_{j \mid t_{(j)} \leq t}{
        \!\!
        \widehat{S}(t_{(j-1)})
        \frac{\bar{D}(j, r)}{\bar{A}(j)}
    }
\end{equation}

\section{Time-to-Event Prediction}

Up until now, 
I have outlined various concepts foundational to survival analysis,
focusing primarily on quantities and statistics of time-to-event outcomes
at a popoulation level.
These measures play an important role in understanding 
and interpretation of survival data.

In the context of precision medicine, however,
the emphasis shifts towards making individualized predictions
taking distinct patient-level characteristics into account.
Consequently, as described in \cref{chap:ml-and-nn}, 
the primary concern lies in 
making accurate predictions on unseen data,
rather than in the exploration of disease etiology and underlying mechanisms.

For prediction of time-to-event outcomes, classical approaches 
include models based on the previously presented semi-parametric Cox model 
as well as various parametric survival models, 
such as those based on exponential, Weibull, or log-normal distributions.
~\autocite{kleinSurvival2003}
This thesis, however, 
explores the use of contemporary machine learning methods 
in time-to-event prediction,
with a particular emphasis on the application of neural networks.

\subsection{Neural Networks and Time-to-Event Outcomes}

The first application of neural networks for time-to-event prediction
was demonstrated by
\citeauthor{faraggiNeural1995} in
\citeyear{faraggiNeural1995},
and involves parameterising the parametric part of the Cox model
with a neural network, 
such that the \(g(\vec{x})\) term in \cref{eq:cox} is a 
flexible neural network model instead of a simple linear function.
\autocite{faraggiNeural1995}

\vspace{.5em}
\begin{equation*}
    \widehat{\lambda} (t \,|\, \vec{x}) = \hzt \exp [
    \eqnmarkbox[color2]{node1}{
        g(\vec{x})
    }]
\end{equation*}
\annotate[yshift=.6em]{left}{node1}{use neural network}

This approach was later further refined
in the \emph{DeepSurv} paper from 
\citeyear{katzmanDeepSurv2018a},
in which modern neural network techniques
were added to Faraggi-Simon framework, 
which markedly improved its usefulness.
~\autocite{katzmanDeepSurv2018a}
\citeauthor{katzmanDeepSurv2018a} showed that the flexibility 
offered by neural networks led to increased performance
in both synthetic and real-life time-to-event prediction applications
compared to a standard Cox model.
However, the \emph{DeepSurv} approach is still limited by the 
assumption of proportional hazards.

\subsection{Overview of Approaches}

Recently, there have been considerable interest in neural network-based
time-to-event prediction models, and as a consequence, many new methods 
have since been developed.
For a thorough overview of the existing approaches, 
\textcite{wiegrebeDeep2023} and 
\textcite{kvammeContinuous2021} provide valuable insights.
Generally, two prevailing types of approaches exists:
continuous-time methods based on the Cox model, 
which includes \emph{DeepSurv},
and discrete-time methods as exemplified by 
\textcite{leeDeepHit2018} and \textcite{gensheimerScalable2019}.

The discrete-time approaches offer several advantages that 
make them particularly relevant for neural network. 
Furthermore, they have been shown to offer better predictive
performance compared to the Cox-based methods.
~\autocite{kvammeContinuous2021, leeDeepHit2018, gensheimerScalable2019}
Notably, \emph{DeepHit}\autocite{leeDeepHit2018}
and \emph{Logistic-Hazard}\autocite{gensheimerScalable2019}
are the two most cited papers in this context as of the time of writing.
Among these and other tested approaches,
\textcite{kvammeContinuous2021} found that 
DeepHit offers excellent discrimination but suffers from poor calibration.
In contrast, the Logistic-Hazard model have nearly as good discrimination
and also significantly better calibration. 
Consequently, the Logistic-Hazard model, and an extension hereof, 
was chosen for application in 
\studyii{} and \studyiii{}.

In the following section, I will be giving a brief description of
this discrete-time formulation of time-to-event analysis and 
elaborate on the Logistic-Hazard model in more detail.

\section{Discrete-Time Survival Analysis}

Most textbooks on survival analysis treats survival time as continuous, 
and that is also usually the case across the biomedical litterature.
However, handling time as a something discrete can be advantegous.
In practice, most measurements of time is inherently discrete 
with durations being recorded in, for example, days; months; and weeks.
The continuous time approaches presented earlier in this chapter, 
are also applicable to discrete time data,
however, methods designed specifically for discrete time-to-event 
data have some advantages~\autocite{tutzModeling2016}:

\begin{itemize}
    \item If observed event times are inherently discrete, 
        then modelling them as such is arguably more appropriate. 
    \item In the discrete-time setting, hazards can be formulated as 
        conditional probabilities which are much more intuitive to 
        both interpret and understand.
    \item Discrete time-to-event models are more easily transferred to 
        other more general purpose modelling frameworks 
        such as generalized linear models, random survival forests, 
        neural networks.
\end{itemize}

The latter point is the main motivation behind both 
the \emph{DeepHit} and \emph{Logistic-Hazard} approach.
For a complete overview of the theory enabling these two approaches,
the book by \textcite{tutzModeling2016} is a valuable resource, 
and serves as the main source of reference for the following.

\subsection{Notation and Definitions}

In the discrete-time framework, 
continuous follow-up time \(\Tic\) is divided into \(q\) contiguous intervals,
that is
%
\begin{equation*}
	(0, a_1], (a_1, a_2], \dots, (a_{q-1}, a_q]
\end{equation*}
%
and \(\Tid \in \{1, \dots, q\}\) is a discrete random  variable
such that if \(\Tid = \tid\) is observed, then the event 
falls in the interval \((a_{\tid-1}, a_{\tid}]\).
Similarly, the discretized censoring time is \(\Cid \in \{1, \dots, q\}\).

With this discrete time scale, 
the distribution of \(\Tid\), 
given some vector of covariates \(\vec{x}\),
can be described using discrete equivalents of the previously 
outlined basic quantities of survival analysis, that is
%
\begin{align}
    \text{probability mass function:} \qquad
    f(\tid \giv \vec{x}) 
    &= \PR (\Tid = \tid \mid \vec{x}) \\
    %
    \text{cumulative mass function:} \qquad
    F(\tid \giv \vec{x}) 
    &= \PR (\Tid \leq \tid \mid \vec{x}) \\
    %
    \label{eq:discrete-hazard}
    \text{hazard function:} \qquad
    \lambda(\tid \giv \vec{x}) 
    &= \PR (\Tid = \tid \mid \Tid \geq \tid, \vec{x}) \\
    %
    \text{survival function:} \qquad
    S(\tid \giv \vec{x}) 
    &= \PR (\Tid > \tid \mid \vec{x})
\end{align}

\subsection{The Logistic-Hazard Model}

\Citeauthor{gensheimerScalable2019}'s approach, 
which they refer to as \enquote{Nnet-survival},
~\autocite{gensheimerScalable2019}
is more accurately characterized as the Logistic-Hazard method, 
as described in \textcite{kvammeContinuous2021}.
In the Logistic-Hazard method, 
the time-to-event data is described by 
modelling the effect of covariates on the discrete hazard function 
(\cref{eq:discrete-hazard}) 
using a neural network.
The concept is not novel, 
employing the discrete hazard for statistical modeling is a common method, 
as covered extensively in \textcite{tutzModeling2016}. 
In addition, a neural-network based model with the same general idea 
was presented by \citeauthor{brownUse1997} in 1997.
~\autocite{brownUse1997}
However, 
\citeauthor{gensheimerScalable2019} were the first to adapt the approach
to current neural network methodologies.

\subsection{Log-Likelihood of the Discrete Hazard}

Let \(\mathfrak{D}_{\mathrm{d}}\) 
be a discrete-time survival dataset of size \(N\),
\begin{equation}
    \mathfrak{D}_{\mathrm{d}} = 
    \{(\tid_i, \sigma_i, \vec{x}_i) \mid i = 1, \ldots, N\},
\end{equation}
where \(\tid_i\) is the discretized survival time, 
\(\sigma_i\) is the event indicator as defined in \cref{eq:sigma-def},
and \(\vec{x}_i = (x_1, x_2, \dots, x_p)'\) is the feature vector.
With the assumption of \emph{noninformative censoring},
~\autocite{kalbfleischStatistical2002}
in the Logistic-Hazard model, 
the contribution of the \(i\)th individual to the likelihood function
can be shown to be  
~\autocite{tutzModeling2016}
\begin{equation}
    \Lik_i = %
    \begin{cases}
        \PR (\Tid_i = \tid_i) & \text{if non-censored} \\
        \PR (\Tid_i > \tid_i) & \text{if censored}.
    \end{cases}
\end{equation}

These two probabilities can be expressed using the discrete hazards,
as it can be seen that
\begin{align}
    \begin{split}
    \PR (\Tid = \tid) 
    &= \PR (\Tid = \tid \mid \Tid \geq \tid) \PR (\Tid  \geq \tid) \\
    &= \PR (\Tid = \tid \mid \Tid \geq \tid) \PR (\Tid  > \tid - 1) \\
    &= \lambda (\tid) \, \prod_{s=1}^{\tid - 1} (1 - \lambda(s))
    \end{split} \\
    \intertext{and similarly}
    \begin{split}
    \PR (\Tid > \tid) 
        &= \PR (\Tid > \tid \mid \Tid \geq \tid) \PR (\Tid  \geq \tid) \\
        &= (1 - \PR (\Tid = \tid \mid \Tid \geq \tid)) \PR (\Tid  \geq \tid) \\
        &= (1 - \PR (\Tid = \tid \mid \Tid \geq \tid)) \PR (\Tid  > \tid - 1) \\
        &= \prod_{s=1}^{\tid} (1 - \lambda(i)).
    \end{split}
\end{align}

Now, by introducing an indicator function, 
defined according to \textcite{tutzModeling2016} as 
\begin{equation}
    \label{eq:lh-indicator}
    \bar{y}_i(\tid) = \begin{cases}
        1, & \text{if individual fails in \((a_{\tid-1}, a_{\tid}]\),} \\
        0, & \text{if individual survives \((a_{\tid-1}, a_{\tid}]\),}
    \end{cases}
\end{equation}
and by including the discrete hazard function and the covariates, 
the likelihood contribution for the \(i\)th individual can be expressed as
\begin{equation}
    \label{eq:lh-likelihood}
    \Lik_i = \prod_{s=1}^{\tid_i} 
        \lambda(s \giv \vec{x}_i)^{\bar{y}_i(s)}
        (1 - \lambda(s \giv \vec{x}_i))^{1 - \bar{y}_i(s)}.
\end{equation}

The total log-likelihood of all datapoints then gives the loss-function
used in the Logistic-Hazard model, 
~\autocite{gensheimerScalable2019, tutzModeling2016}
which can be expressed
\begin{equation}
    \label{eq:lh-loglikelihood}
    \lik = 
        \sum_{i = 1}^{N} 
            \sum_{s=1}^{\tid_i} 
                \bar{y}_i(s) \log (\lambda(s\giv\vec{x}_i))
                + (1 - \bar{y}_i(s)) \log (1 - \lambda(s\giv\vec{x}_i)).
\end{equation}

\subsection{Loss Function Explained}

\def\y#1#2{\hat{\lambda}_{#1#2}}
\def\yy#1#2{1\!-\!\y{#1}{#2}}

As an example, the following set of observations with discretized 
time-to-event data with a single risk, e.g. all-cause mortality, 
and a follow-up time that have been discretized into seven contiguous intervals,
constitutes a survival dataset.

\begin{equation}
\begin{tabular}{r  ccccc}
    \toprule
    subject   \(i\)      & 1 & 2 & 3 & 4 & 5 \\
    \midrule
    time    \(\tau_i\)   & 5 & 7 & 4 & 5 & 3 \\
    event   \(\sigma_i\) & 1 & 1 & 0 & 0 & 1 \\
    \bottomrule
\end{tabular}
\end{equation}

In this setting,  using the Logistic-Hazard model, 
the neural network output for this dataset is a 2-dimensional
matrix with \(5\) rows  (subjects) and \(7\) columns (time intervals),
and each entry is the predicted conditional hazard for the
specific subject at a specific timepoint. We can write this as
\begin{equation}
\hat{\bm{\Lambda}}= \!\!\!\!
\begin{tikzpicture}
[   on grid,
    font = \small,
    baseline = -.7ex,
    inner sep=1pt,
    outer sep=0pt,
    minimum width=9mm,
    minimum height=6mm,
    every left delimiter/.style={xshift=2.5ex},
    every right delimiter/.style={xshift=-3.0ex}
]
\matrix (pred) [
	matrix of math nodes, 
    left delimiter={[}, 
    right delimiter={]},
]{ 
\y{1}{1} & \y{1}{2} & \y{1}{3} & \y{1}{4} & \y{1}{5} & \y{1}{6} & \y{1}{7} \\
\y{2}{1} & \y{2}{2} & \y{2}{3} & \y{2}{4} & \y{2}{5} & \y{2}{6} & \y{2}{7} \\
\y{3}{1} & \y{3}{2} & \y{3}{3} & \y{3}{4} & \y{3}{5} & \y{3}{6} & \y{3}{7} \\
\y{4}{1} & \y{4}{2} & \y{4}{3} & \y{4}{4} & \y{4}{5} & \y{4}{6} & \y{4}{7} \\
\y{5}{1} & \y{5}{2} & \y{5}{3} & \y{5}{4} & \y{5}{5} & \y{5}{6} & \y{5}{7} \\
};
\useasboundingbox[anchor=center] (pred.north west) rectangle (pred.south east);
\draw[->, black!50] ([xshift=2ex] pred-1-7.north east) -- ([xshift=2ex] pred-2-7.east)
    node [midway, font=\scriptsize, above, sloped] {subject};
\draw[->, black!50] ([yshift=1ex] pred-1-6.north) -- ([yshift=1ex] pred-1-7.north)
    node [midway, font=\scriptsize, above] {time};
\end{tikzpicture}
\end{equation}

Now, the indicator function can be computed 
using the defintion in \cref{eq:lh-indicator}
and the observed data \(\tid_i\) and \(\sigma_i\). 
In matrix form, the output of this function is
\begin{equation}
\bar{\bm{Y}}= \!\!\!\!
\begin{tikzpicture}
[   on grid,
    font = \small,
    baseline = -.7ex,
    inner sep=1pt,
    outer sep=0pt,
    minimum width=9mm,
    minimum height=6mm,
    every left delimiter/.style={xshift=2.5ex},
    every right delimiter/.style={xshift=-3.0ex}
]

\matrix (mask) [
	matrix of math nodes, 
    left delimiter={[}, 
    right delimiter={]}
]{ 
0 & 0 & 0 & 0 & 1 & - & - \\
0 & 0 & 0 & 0 & 0 & 0 & 1 \\
0 & 0 & 0 & 0 & - & - & - \\
0 & 0 & 0 & 0 & 0 & - & - \\
0 & 0 & 1 & - & - & - & - \\
};
\useasboundingbox[anchor=center] (mask.north west) rectangle (mask.south east);
\end{tikzpicture}
\end{equation}

Now, combining these two matrices according to the formula
in \cref{eq:lh-likelihood}, we obtain the likelihood in matrix form as
\begin{equation}
    \mathcal{L}(\bm{\hat{\Lambda}}, \bm{\bar{Y}}) = \!\!\!\!
\begin{tikzpicture}
[   on grid,
    font = \small,
    baseline = -.7ex,
    inner sep=1pt,
    outer sep=0pt,
    minimum width=9mm,
    minimum height=6mm,
    every left delimiter/.style={xshift=2.5ex},
    every right delimiter/.style={xshift=-3.0ex}
]
\matrix (loss) [
	matrix of math nodes, 
    left delimiter={[}, 
    right delimiter={]},
    nodes={font=\footnotesize}
]{ 
\yy{1}{1} & \yy{1}{2} & \yy{1}{3} & \yy{1}{4} & \y{1}{5}  & -         & -         \\
\yy{2}{1} & \yy{2}{2} & \yy{2}{3} & \yy{2}{4} & \yy{2}{5} & \yy{2}{6} &  \y{2}{7} \\
\yy{3}{1} & \yy{3}{2} & \yy{3}{3} & \yy{3}{4} & -         & -         & -         \\
\yy{4}{1} & \yy{4}{2} & \yy{4}{3} & \yy{4}{4} & \yy{4}{5} & -         & -         \\
\yy{5}{1} & \yy{5}{2} & \y{5}{3}  & -         & -         & -         & -         \\
};
\useasboundingbox[anchor=center] (loss.north west) rectangle (loss.south east);
\end{tikzpicture}
\end{equation}
from which the log-likelihood, \cref{eq:lh-loglikelihood}, is then
\begin{equation*}
    \small
\begin{split}
    \mathscr{l}(\bm{\hat{\Lambda}}, \bm{\bar{Y}})
    &={} \log (\yy{1}{1}) + \log(\yy{1}{2}) + \log(\yy{1}{3}) + \log(\yy{1}{4}) 
     +  \log(\y{1}{5}) \\ 
    &+{} \log (\yy{2}{1}) + \log(\yy{2}{2}) + \log(\yy{2}{3}) + \log(\yy{2}{4}) 
     +  \log(\yy{2}{5}) + \log(\yy{2}{6}) +  \log(\y{2}{7}) \\
    &+{} \log (\yy{3}{1}) + \log(\yy{3}{2}) + \log(\yy{3}{3}) + \log(\yy{3}{4}) \\
    &+{} \log (\yy{4}{1}) + \log(\yy{4}{2}) + \log(\yy{4}{3}) + \log(\yy{4}{4}) 
    + \log(\yy{4}{5})  \\
    &+{} \log (\yy{5}{1}) + \log(\yy{5}{2}) + \log(\y{5}{3} )
\end{split}
\end{equation*}
