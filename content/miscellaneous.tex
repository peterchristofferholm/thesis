
\chapter*{Misc}

A specific chromosomal translocation in 
chronic myeloid leukemia (CML) 
results in a constitutively active tyrosine-kinase (BCR-ABL)
that can be targeted with tyrosine kinase inhibitors such as imatinib~%
\autocite{drukerEfficacy2001}.

%------------------------------------------------------------------------
\vskip 5em

For predictive models to be effective,
they must enable clinicians to guide treatment by:
%
\begin{enumerate*}
    \item providing actionable insights on modifiable risk factors
    \item being adapted to the individual patient
    \item being generally applicable in different populations
\end{enumerate*}

%------------------------------------------------------------------------
\vskip 5em

AI-driven medical prediction models are able to synthesize large amounts
of electronic health data into risk scores or estimates that can be used
to assess the risk and condition of the individual patient~\autocite{handelmanEDoctor2018}.

%------------------------------------------------------------------------
\vskip 5em

\begin{itemize}
    \item Draw illustration of overfitting. Some sort of classifier with a
        super irregular decision boundary.
\end{itemize}

%------------------------------------------------------------------------
\vskip 5em


Complications also occur in otherwise asymptomatic patients, 
and for that reason risk evaluation should apply to both patients
with out without symptoms.
~\autocite{knuuti20192020}

Although revascularization can relieve symptoms for the culprit lesions
responsible for myocardial ischemia, it may not prevent the progression
of other neighboring lesions and future acute thrombotic events.
~\autocite{libbyPathophysiology2005}

\textquote[byrne20232023]{%
    A number of prognostic models \textelp{}
    have been formulated into clinical risk scores and, among
    these, the GRACE risk score offers the best discriminative performance and
    is therefore recommended for risk assessment.%
}

\textquote[knuuti20192020]{% class I recommendation
    Invasive coronary angiography is recommended as an alternative test to
    diagnose \textsc{CAD} \textins{coronary artery disease} in patients with a high
    clinical likelihood, severe symptoms refractory to medical therapy or
    typical angina at a low level of exercise, and clinical evaluation that
    indicates high event risk.%
}

\textquote[knuuti20192020]{% class IIa recommendation
    Invasive coronary angiography with the availability of invasive functional
    evaluation should be considered for confirmation of the diagnosis of CAD in
    patients with an uncertain diagnosis on non-invasive testing%
}

Myocardial tissue damage can be detected with high sensitivity by measuring 
the release troponins and creatinine-kinase from the sarcolemma.
~\autocite{falkPathogenesis2006}

The esc guidelines on risk-prevention highlights 
that one of the currents gaps in evidence,
is comparing the performance of competing risk-adjusted 
cardiovascular disease risk models versus the standard models.
~\autocite{visseren20212021}
