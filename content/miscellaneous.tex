
\chapter*{Misc}


%------------------------------------------------------------------------
\vskip 5em

For predictive models to be effective,
they must enable clinicians to guide treatment by:
%
\begin{enumerate*}
    \item providing actionable insights on modifiable risk factors
    \item being adapted to the individual patient
    \item being generally applicable in different populations
\end{enumerate*}

%------------------------------------------------------------------------
\vskip 5em

AI-driven medical prediction models are able to synthesize large amounts
of electronic health data into risk scores or estimates that can be used
to assess the risk and condition of the individual patient~\autocite{handelmanEDoctor2018}.

%------------------------------------------------------------------------
\vskip 5em

\begin{itemize}
    \item Draw illustration of overfitting. Some sort of classifier with a
        super irregular decision boundary.
\end{itemize}

%------------------------------------------------------------------------
\vskip 5em


Complications also occur in otherwise asymptomatic patients, 
and for that reason risk evaluation should apply to both patients
with out without symptoms.
~\autocite{knuuti20192020}

Although revascularization can relieve symptoms for the culprit lesions
responsible for myocardial ischemia, it may not prevent the progression
of other neighboring lesions and future acute thrombotic events.
~\autocite{libbyPathophysiology2005}

\textquote[byrne20232023]{%
    A number of prognostic models \textelp{}
    have been formulated into clinical risk scores and, among
    these, the GRACE risk score offers the best discriminative performance and
    is therefore recommended for risk assessment.%
}

\textquote[knuuti20192020]{% class I recommendation
    Invasive coronary angiography is recommended as an alternative test to
    diagnose \textsc{CAD} \textins{coronary artery disease} in patients with a high
    clinical likelihood, severe symptoms refractory to medical therapy or
    typical angina at a low level of exercise, and clinical evaluation that
    indicates high event risk.%
}

\textquote[knuuti20192020]{% class IIa recommendation
    Invasive coronary angiography with the availability of invasive functional
    evaluation should be considered for confirmation of the diagnosis of CAD in
    patients with an uncertain diagnosis on non-invasive testing%
}

Myocardial tissue damage can be detected with high sensitivity by measuring 
the release troponins and creatinine-kinase from the sarcolemma.
~\autocite{falkPathogenesis2006}

The esc guidelines on risk-prevention highlights 
that one of the currents gaps in evidence,
is comparing the performance of competing risk-adjusted 
cardiovascular disease risk models versus the standard models.
~\autocite{visseren20212021}




\section{Health Informatics}

Health informatics is a cross-disciplinary field
that aims at developing methods and techniques
for the collection, study, and analysis of healthcare data.
It operates in the intersection between medicine and computer science,
and involves disciplines such as
bioinformatics, software engineering, statistics, information systems,
data science, and artificial intelligence.
The overall goal of health informatics is the data-driven improvement
of delivery and accuracy of health and healthcare for the individual patient.

\section{Precision Medicine is a Process}

In their review 
\citetitle{konigWhat2017}, 
\citeauthor{konigWhat2017} defines precision medicine as 
a process with three partly sequential \enquote{tracks},
which serves as an useful abstraction or framework 
for describing the process involved in developing precision medicine.
~\autocite{konigWhat2017} 



Central to precision medicine, and also their presented framework,
is, of course, \enquote{the data}. 
Now, the three tracks are
1) preprocessing and data-mining
2) diagnostic and prognostic models
and 3) prediction of treatment response.





\subsection{Track 1: preprocessing and data mining}

Quality control, preprocessing, and extraction of information.
The concrete steps and methods utilized as part of this track
is highly context-dependent, 
and is therefore difficult to describe on a general level. 

Feature-engineering.

Extracting structured information from unstructured information.

Clustering.

\subsection{Track 2: diagnostic and prognostic models}

Relying in part on processed data from track 1
as well as any insights gained from data-mining,
track 2 is concerned with the development
of models that can characterise current (diagnostic models)
or future (prognostic models) states of health and disease~%
\autocite{konigWhat2017}. 

\subsection{Track 3: predicting treatment response}

Biomedical informatics.

By considering deep phenotypic characteristics including 
biomarker profiles, genetics, clinical history, etc. 

Finding determinants of treatment response

Arguably, the greatest advances have come from genetics, especially in oncology.
Today, patients with a cancer diagnosis routinely undergo molecular testing
to identify the ideal course of treatment.

Electronic health records make up an extremely rich resource of clinical
information that have the possibility to fuel the development of the 
precision medicine of tomorrow.






Use data to improve decision making

Subgroup identification seeks to identify groups of individuals with an
increased treatment response, which
\citeauthor{kosorokPrecision2019} refers to as 
\textquote[kosorokPrecision2019]{%
    finding the right patient for the right treatment%
}.


Genomic medicine

Electronical health records are a rich source of health data,
and can be used to find connections between risk factors and diseases.

Precision medicine is prevention and treatment approaches

that takes individual variation into account.


The terms
\enquote{precision},
\enquote{personalized},
and \enquote{individualized medicine}
are often used interchangeably.

avoid toxicity and improve efficacy

\section{Deep phenotyping} \label{deep-pheno}


\todo[inline, caption={Some text}]{
\begin{minipage}{\linewidth}
    \emph{Notes:}

    Causality.
    Different strategies exists for guiding the development of 
    predicting therapy response.
    We might be able to use prognostic and diagnostic factors
    discovered in the previous track to predict treatment response.
    Example: HER is prognostic for survival in breast cancer,
        but is also predictive of treatment response against X.

\end{minipage}
}

As with the other tracks, track 3 gan generate further knowledge 
that in turn can feed back to the more general phenotyping of patients
\autocite{konigWhat2017}. 


In modern medicine, 
ever increasing amounts of data 
is continuously being generated and collected.
Ranging from structured administrative data 
used primarily for billing purposes
to advanced imaging and high-througput \enquote{omics} analyses,
the array of available data is as diverse as it is plentiful.

The volume and range of information collected
at just a single visit for a single patient
can be large.

An electronic health record
is a systematized collection of health data 
stored in a digital format.
Electronic health records may include data such as
medical history, drug prescriptions, laboratory test results,
radiology images, body measurements, and billing information.
EHRs enable following patients in both time and space;
not only in the physical space that is the cardiology department
and the intensive care unit, but also in the health-and-disease space.

\marginnote{
    The five Vs of big data:
    \begin{itemize}
        \item Volume
        \item Velocity
        \item Variety
        \item Value
        \item Veracity
    \end{itemize}
}

\section{Clinical decision support systems}

Deep learning can identify patterns in sparse, noisy,
and very heterogenous data without the need for expert feature engineering
\cite{norgeotCall2019}.

Challenges in developing machine learning models from electronic health records
includes syntactic interoperability, 
which specifies the format and structure of the data.
Many different interopability formats exists, 
but their adoption varies considerably.

Benefits of interopable digital health data
~\autocite{lehneWhy2019}.
Test~\cite{lehneWhy2019}.

\begin{itemize}
    \item Large-scale observational studies.
    \item Less time spend on data cleaning and pre-processing.
    \item Interoperable exchange format could further cross-institutional
    and international collaboration, which would make it easier 
    to reproduce and externally validate e.g. prognostic applications.
    In the case of rare diseases with very limited number of patients, 
    international pooling of data could enable analysis
    that otherwise would not be possible.
    \item Faster research and development process.
    \item If data is known to conform to a well-defined format,
    computer software can be written without explicit access to the data.
    This solve many issues related to sensitive data or
    data that are otherwise subject to strict data protection regulations.
\end{itemize}


Healthcare data are usually private and scattered across various applications,
which makes it difficult to share data and generate robust results 
that translate to different and diverse populations.


Succesful implementation of data-driven applications
require large and diverse data sets.

AI approaches rely on data that accurately represent
the real-life distribution of the underlying problem.
We can not exclusively rely on data that have been carefully curated 
from few and often very similar sources. 
In order to capture subtle relationships 
between health and disease patterns,
we need to include many and diverse cases~%
\autocite{riekeFuture2020}.

Federated learning could solve some of these issues\autocite{riekeFuture2020}.
