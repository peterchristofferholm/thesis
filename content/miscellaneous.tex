
\chapter*{Misc}

For predictive models to be effective,
they must enable clinicians to guide treatment by:
%
\begin{enumerate*}
    \item providing actionable insights on modifiable risk factors
    \item being adapted to the individual patient
    \item being generally applicable in different populations
\end{enumerate*}

%------------------------------------------------------------------------
\vskip 5em

AI-driven medical prediction models are able to synthesize large amounts
of electronic health data into risk scores or estimates that can be used
to assess the risk and condition of the individual patient~\autocite{handelmanEDoctor2018}.

Complications also occur in otherwise asymptomatic patients, 
and for that reason risk evaluation should apply to both patients
with out without symptoms.
~\autocite{knuuti20192020}

Although revascularization can relieve symptoms for the culprit lesions
responsible for myocardial ischemia, it may not prevent the progression
of other neighboring lesions and future acute thrombotic events.
~\autocite{libbyPathophysiology2005}

\textquote[byrne20232023]{%
    A number of prognostic models \textelp{}
    have been formulated into clinical risk scores and, among
    these, the GRACE risk score offers the best discriminative performance and
    is therefore recommended for risk assessment.%
}

\textquote[knuuti20192020]{% class I recommendation
    Invasive coronary angiography is recommended as an alternative test to
    diagnose \textsc{CAD} \textins{coronary artery disease} in patients with a high
    clinical likelihood, severe symptoms refractory to medical therapy or
    typical angina at a low level of exercise, and clinical evaluation that
    indicates high event risk.%
}

\textquote[knuuti20192020]{% class IIa recommendation
    Invasive coronary angiography with the availability of invasive functional
    evaluation should be considered for confirmation of the diagnosis of CAD in
    patients with an uncertain diagnosis on non-invasive testing%
}

Myocardial tissue damage can be detected with high sensitivity by measuring 
the release troponins and creatinine-kinase from the sarcolemma.
~\autocite{falkPathogenesis2006}

The esc guidelines on risk-prevention highlights 
that one of the currents gaps in evidence,
is comparing the performance of competing risk-adjusted 
cardiovascular disease risk models versus the standard models.
~\autocite{visseren20212021}


Subgroup identification seeks to identify groups of individuals with an
increased treatment response, which
\citeauthor{kosorokPrecision2019} refers to as 
\textquote[kosorokPrecision2019]{%
    finding the right patient for the right treatment%
}.


Electronical health records are a rich source of health data,
and can be used to find connections between risk factors and diseases.

Precision medicine is prevention and treatment approaches

that takes individual variation into account.


The terms
\enquote{precision},
\enquote{personalized},
and \enquote{individualized medicine}
are often used interchangeably.

avoid toxicity and improve efficacy

\todo[inline, caption={Some text}]{
\begin{minipage}{\linewidth}
    \emph{Notes:}
    Causality.
    Different strategies exists for guiding the development of 
    predicting therapy response.
    We might be able to use prognostic and diagnostic factors
    discovered in the previous track to predict treatment response.
    Example: HER is prognostic for survival in breast cancer,
        but is also predictive of treatment response against X.

\end{minipage}
}


As with the other tracks, track 3 gan generate further knowledge 
that in turn can feed back to the more general phenotyping of patients
\autocite{konigWhat2017}. 

An electronic health record
is a systematized collection of health data 
stored in a digital format.
Electronic health records may include data such as
medical history, drug prescriptions, laboratory test results,
radiology images, body measurements, and billing information.
EHRs enable following patients in both time and space;
not only in the physical space that is the cardiology department
and the intensive care unit, but also in the health-and-disease space.

\marginnote{
    The five Vs of big data:
    \begin{itemize}
        \item Volume
        \item Velocity
        \item Variety
        \item Value
        \item Veracity
    \end{itemize}
}


\section{Clinical decision support systems}

Challenges in developing machine learning models from electronic health records
includes syntactic interoperability, 
which specifies the format and structure of the data.
Many different interopability formats exists, 
but their adoption varies considerably.

Benefits of interopable digital health data
~\autocite{lehneWhy2019}.
Test~\cite{lehneWhy2019}.

\begin{itemize}
    \item Large-scale observational studies.
    \item Less time spend on data cleaning and pre-processing.
    \item Interoperable exchange format could further cross-institutional
    and international collaboration, which would make it easier 
    to reproduce and externally validate e.g. prognostic applications.
    In the case of rare diseases with very limited number of patients, 
    international pooling of data could enable analysis
    that otherwise would not be possible.
    \item Faster research and development process.
    \item If data is known to conform to a well-defined format,
    computer software can be written without explicit access to the data.
    This solve many issues related to sensitive data or
    data that are otherwise subject to strict data protection regulations.
\end{itemize}


Healthcare data are usually private and scattered across various applications,
which makes it difficult to share data and generate robust results 
that translate to different and diverse populations.

AI approaches rely on data that accurately represent
the real-life distribution of the underlying problem.
We can not exclusively rely on data that have been carefully curated 
from few and often very similar sources. 
In order to capture subtle relationships 
between health and disease patterns,
we need to include many and diverse cases~%
\autocite{riekeFuture2020}.

Automated diagnosis in ophthalmology 
from optical coherence tomography scans~\autocite{defauwClinically2018}.

\section{Computational models in cardiology}

Using single-lead lectrocardiogram recordings from \num{53877} patients,
Hannun et al.\autocite{hannunCardiologistlevel2019}
constructed a deep neural network
to identify 12 difference classes of cardiac rythm.
In the validation of their algorithm on an independent test set,
the authors were able to show that the computer model
significantly outperforms the average cardiologist. 


%%%%%%%%%%%%%%%%%%%%%%%%%%%%%%%%%%%%%%%%%%%%%%%%%%%%%%%%%%%%%%%%%%%%%%%%%%%%%%%

\section{Miscellaneous}

Systematic deugging and continuous monitoring and validation 
is of utmost importance if we are to release AI algorithms into the wild%
\autocite{topolHighperformance2019}.

In computer vision tasks in the medical domain,
deep-learning models have achieved physician-level performance
in many different diagnostic tasks
ranging from \todo{finish sentence}.


Deep learning is at its core a form of representation learning.
~\autocite{estevaGuide2019}
Each layer in a neural network is a different representation,
and by stacking several of such layers on top of each others,
the representation in one hidden layer
feeds into the next layer and
is thereby being transformed into an even more abstract representation%
~\autocite{estevaGuide2019}.
This allow neural network models to identify patterns in sparse, noisy,
and highly heterogeneous data without the need for expert feature engineering,
which makes them particularly pertinent to healthcare applications.% 
~\sidecite[-2em]{norgeotCall2019}



\subsection{Scope of Explanations}

The scope of an explanation is the difference between
explanations for a complete model and
explanations for a single output.
Global explanation covers feature importance estimates 
for the entire dataset.
Local explanations, on the other hand, seeks to explain
the impact of the specific example under scrutiny.

A SHAP-waterfall plot is an example of a local explanation.
A saliency map of a chest radiograph that shows
which pixels contributed to the label \enquote{liver cancer}
is another example.

Shapley values measures the marginal contribution
of each individual feature.


One limitation of XAI models is the accuracy and relevance of explanations.
Explainability algorithms such as SHAP are only approximations
of the complete model.
In other words, the fidelity is not perfect and therefore neither
is the explanation.
However, for black-box models such as neural networks,
it is the next best thing.

