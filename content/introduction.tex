\chapter{Thesis Objectives and Overview} \label{intro}

The primary objective of this thesis is to contribute to the advancement of
precision medicine in the field of ischemic heart disease, with a particular
emphasis on secondary prevention. The research focuses on two main areas: 1)
characterizing multimorbidity in ischemic heart disease and its impact on
disease risk and progression, and 2) developing risk-prediction tools capable
of leveraging heterogeneous healthcare data. For the latter, we have employed
neural network-based survival models designed to accurately handle censored
data.

This thesis investigates how data-driven approaches can enhance the evolution
of precision medicine specifically for ischemic heart disease. While the
research is concentrated on ischemic heart disease and cardiology, the
methodologies and findings could potentially be extrapolated to other medical
fields, given their broad applicability.

The intersection of precision medicine and artificial intelligence represents a
significant paradigm shift, offering the potential to revolutionize how we
diagnose, manage, and treat diseases.

\section{Overview of Structure}

The thesis is organized as follows:

\begin{itemize} 

    \item In the chapter \nameref{ihd-background}, I provide an
        in-depth discussion of the pathophysiology and disease manifestations of
        ischemic heart disease, which serves as the focal point of this thesis.

    \item Subsequently, in the chapter \nameref{precision-medicine}, I offer an
        overview of the methodologies involved in developing data-driven
        precision medicine. Throughout the thesis, precision medicine is
        primarily contextualized within the frameworks of \enquote{big data} 
        and \enquote{machine learning}.

    \item The chapter \nameref{ml-fundamentals} introduces the core concepts of
        machine learning, with a specific focus on neural networks. While the
        chapter is generally broad in scope, it places particular emphasis on
        the tools and techniques utilized in the research presented.

    \item The final background chapter, \nameref{survival-analysis}, provides a
        comprehensive overview of survival analysis. It further delves into the
        specific approach we have employed for modeling time-to-event data
        using neural networks.  

    \item In \nameref{results}, I summarize each of the three included papers,
        briefly outline the methodologies employed, and highlight the key
        findings.  Additionally, I contextualize the research within the
        broader scientific literature.

    \item In \nameref{conclusions}, I offer final thoughts on the thesis and
        outline areas that warrant further investigation.

    \item The \nameref{appendix} includes the three full-length scientific
        manuscripts that form the core of this thesis.

\end{itemize}

\todo[inline]{Refactor \enquote{Thesis Overview} section.}
