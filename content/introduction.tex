\chapter{Introduction: Background and Objectives} \label{intro}

With the advent of large language models (LLMs) like ChatGPT
%------------------------------------------------------------------------------
\marginnote{%
    \textsf{hype}, (noun). [clipping of hyperbole].

    (marketing) promotion or propaganda; especially exaggerated claims. 
    \rightline{\textit{wiktionary.org}}%
}
%------------------------------------------------------------------------------
and text-to-image models such as DALL-E and Midjourney,
there has been a surge of interest in the broader domains of
artificial intelligence and machine-learning.
%
ChatGPT, in particular, stands out as prominent example---%
an LLM-based chatbot developed by OpenAI, 
have managed to impress both the general public aswell 
as researchers across various disciplines.
It has been shown that ChatGPT can pass exams
such as USMLE\footnotemark
~\autocite{openaiGPT42023}

%------------------------------------------------------------------------------
\footnote{%
The United States Medical Licensing Examination (USMLE) 
is a three-step examination program for obtaining a medical license
in the United States of America.
}
%------------------------------------------------------------------------------






It has been claimed that 




However, 




The convergence of precision medicine and artificial intelligence 
stands as a monumental paradigm shift, promising to redefine
the way we diagnose, treat, and manage disease.



It is this author's opinion, 
that the main cause of \enquote{success} of the abovementioned
AI-applications is the combined value of 
absolutelive massive datasets and massive computing power.

This thesis has little to do with the likes of ChatGPT, 
but 

In this thesis, I explore how data-drive approaches
can further the development of precision medicine in ischemic heart disease.

The thesis is structured as follows:
in \autoref{deep-pheno}.


Although we in the three manuscripts that forms this thesis 
primarily place a focus on cardiology,
and specifically ischemic heart disease,
the presented methods can be used across many fields of medicine.



